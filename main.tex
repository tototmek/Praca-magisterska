%-----------------------------------------------
%  Engineer's & Master's Thesis Template
%  Copyleft by Artur M. Brodzki & Piotr Woźniak
%  Warsaw University of Technology, 2019-2022
%-----------------------------------------------

\documentclass[
    bindingoffset=5mm,  % Binding offset
    footnoteindent=3mm, % Footnote indent
    hyphenation=true    % Hyphenation turn on/off
]{src/wut-thesis}

\graphicspath{{tex/img/}} % Katalog z obrazkami.

%-------------------------------------------------------------
% Wybór wydziału:
%  \facultyeiti: Wydział Elektroniki i Technik Informacyjnych
%  \facultymeil: Wydział Mechaniczny Energetyki i Lotnictwa
% --
% Rodzaj pracy: \EngineerThesis, \MasterThesis
% --
% Wybór języka: \langpol, \langeng
%-------------------------------------------------------------
\facultyeiti    % Wydział Elektroniki i Technik Informacyjnych
\MasterThesis % Praca inżynierska
\langpol % Praca w języku polskim

\usepackage{algorithm}
\usepackage{algpseudocode}
\floatname{algorithm}{Pseudokod}
\usepackage{flafter}
\usepackage{tabularray}
\usepackage{listings}
\usepackage{xcolor}
\usepackage{subcaption}
\usepackage{placeins}
\usepackage{minted}
\usemintedstyle{friendly}

\usepackage{makecell}
\renewcommand\cellgape{\Gape[2pt]}
\renewcommand{\cellalign}{cl}

\addbibresource{bibliografia.bib}

\begin{document}

%------------------
% Strona tytułowa
%------------------
\instytut{Automatyki i Informatyki Stosowanej}
\kierunek{Automatyka i Robotyka}
\title{
    Urządzenie zliczające pszczoły w ulu oparte o zestaw czujników pojemnościowych
}
\author{Tomasz Żebrowski}
\album{310371}
\promotor{prof. dr hab. inż. Paweł D. Domański}
\date{\the\year}
\maketitle

%-------------------------------------
% Streszczenie po polsku dla \langpol
% English abstract if \langeng is set
%-------------------------------------
\cleardoublepage % Zaczynamy od nieparzystej strony
\abstract

W niniejszej pracy przedstawiony został system automatycznego zliczania pszczół na wejściu ula, wykorzystujący szereg czujników pojemnościowych.
Wzrost pojemności elektrycznej kondensatorów, wywołany wysokim współczynnikiem przenikalności elektrycznej owadów, mierzony jest z wykorzystaniem nowatorskiej metody akwizycji, ograniczającej do maksimum złożoność i koszt systemu, zapewniając przy tym jego satysfakcjonującą czułość.
Opracowane zostały metody przetwarzania zebranych danych pozwalające precyzyjnie określić chwile pojawiania się pszczół w czujniku, a także kierunek ich ruchu.
Zaprezentowane zostały dwa algorytmy detekcji: oparty na automacie stanowym wykrywającym sekwencje skoków sygnału, oraz wykorzystujący korelację krzyżową sygnału z wzorcem.
Walidacja rozwiązania, składająca się z testów laboratoryjnych wykorzystujących modele pszczół, a także testów w warunkach rzeczywistych z wykorzystaniem ręcznie anotowanych danych, pozwoliła potwierdzić poprawne działanie stworzonego systemu.
Nadaje się on do szacowania natężenia ruchu zbieraczek na wylotku ula, natomiast stosowalność do aproksymacji rozmiaru kolonii pozostaje ograniczona ze względu na występujące błędy zliczania.
Zbudowane urządzenie stanowi dowód skuteczności opracowanych metod, które w wyniku przyszłego rozwoju mogą przynieść jeszcze lepsze efekty.


\keywords pszczelarstwo, czujnik pojemnościowy, przetwarzanie danych

% %----------------------------------------
% % Streszczenie po angielsku dla \langpol
% % Polish abstract if \langeng is set
% %----------------------------------------
% \clearpage
% \secondabstract \kant[1-3]
% \secondkeywords XXX, XXX, XXX

\pagestyle{plain}

%--------------
% Spis treści
%--------------
\cleardoublepage % Zaczynamy od nieparzystej strony
\tableofcontents

%------------
% Rozdziały
%------------
\cleardoublepage % Zaczynamy od nieparzystej strony
\pagestyle{headings}

\clearpage
\section{Wprowadzenie}
\subsection{Wstęp teoretyczny}

Pszczelarstwo jest zajęciem wymagającym wiele pracy, wiedzy i~doświadczenia.
W dzisiejszych czasach, pszczelarze wspierani są różnorodnym sprzętem, umożliwiającym skuteczniejszą opiekę nad ulami, oraz wydajną produkcję miodu.
Ponadto, przez ostatnie lata rozwijane były nowoczesne technologie, które, wykorzystując elektronikę, pozwalają dogłębnie monitorować stan pszczelich kolonii.
Rosnące w~popularność zaawansowane czujniki zapewniają pszczelarzom precyzyjne dane o~zdrowiu pszczół, pozwalające na skuteczniejsze zarządzanie pasieką -- z~pozytywnymi skutkami zarówno dla owadów, jak i~efektywności samego biznesu.
Stosowane szeroko rozwiązania pozwalają, jak opisują Hadjur \cite{Hadjur2022} i~Danieli \cite{Danieli2024} wraz z~zespołami, na monitorowanie takich parametrów jak: temperatura, wilgotność i~stężenie $\text{CO}_2$ wewnątrz ula -- których nieprawidłowe poziomy mogą powodować choroby; waga całego gniazda, świadcząca o~łącznej ilości pszczół, zebranego miodu, i~innych zasobów.
Stosuje się również czujniki akustyczne, mierzące częstotliwość, intensywność i~barwę wibracji, pozwalające na wykrywanie podwyższonego stresu pszczół, a~także początków rojenia się.

Jednym z~zadań realizowanym przez systemy elektroniczne, wymienianym przez Meiklego i~Holsta \cite{Meikle2014}, jest zliczanie pszczół wchodzących do ula i~opuszczających go -- co pozwala szacować chwilowe natężenie ruchu zbieraczek.
Metryka ta zapewnia informacje odnośnie siły i~struktury wiekowej kolonii, a~także zapotrzebowania i~dostępności pożywienia \cite{McLellan1977}.
Zliczanie pszczół ma również znaczenie podczas prowadzenia badań naukowych, które ich dotyczą -- między innymi prace Kolmesa i~Sama \cite{Kolmes1990}, czy Corbeta z~zespołem \cite{Corbet1993}.

\subsection{Stosowane metody automatycznego zliczania pszczół}

Potrzeba liczenia pszczół prowadziła przez lata do rozwoju metod automatyzujących to zadanie.
Przegląd kolejnych pojawiających się rozwiązań \cite{Odemer2021} otwiera ponad stuletnia praca Lundiego z~1925 roku -- urządzenie elektromechaniczne, w~którym owady przechodzące przez system zapadek aktywowały styki, wysyłając tym samym sygnał do licznika.
Podejście to było awaryjne i~wymagało częstego czyszczenia z~nanoszonego pyłku, jednak pozwoliło autorowi na realizację badań o~życiu pszczół, i~położyło podwaliny pod dalszy rozwój podobnych technik \cite{Lundie1925}.
W późniejszych latach powstawały systemy zliczania pszczół oparte na układach optycznych. 
Wykorzystywano fotokomórki na światło podczerwone, którego promień, niewidzialny dla pszczół, był przez nie przecinany, co generowało sygnał.
Do pierwszych prac bazujących na tej technice należy dzieło Ericksona wraz z~zespołem z~roku 1975 \cite{Erickson1975}.
Globalny rozwój technologii umożliwił dalsze rozpowszechnianie rozwiązań opartych na optyce: systemy wykorzystujące technologię LED zaprezentowali m.in. Pešović \cite{Pesovic2017} i~Jiang \cite{Jiang2016} wraz z~zespołami, a~także, w~ostatnich latach, Cunha z~zespołem \cite{Cunha2020}.
Autorzy zwrócili uwagę na znaczącą wadę takiego podejścia -- zbierający się w~torze optycznym pyłek po pewnym czasie uniemożliwiał poprawne działanie systemu, wymuszając regularne czyszczenie wnętrz urządzeń.
Pomimo tego ograniczenia, urządzenia zliczające pszczoły oparte na technologii LED są dzisiaj dostępne komercyjnie \cite{sklep}.

W ostatnich latach, w~literaturze pojawia się coraz więcej systemów zliczających pszczoły opartych na zaawansowanych algorytmach przetwarzania obrazu i~wizji komputerowej.
Narzędzie wykrywające pszczoły w~obrazie wykorzystujące konwolucyjne sieci neuronowe zostało zaprezentowane przez Tauscha wraz z~zespołem \cite{Tausch2020}.
Takie rozwiązania są znacznie bardziej wymagające obliczeniowo, jednak mogą pozwalać na realizację dodatkowych funkcji -- system zaprezentowany przez Bilika wraz z~zespołem \cite{Bilik2021} umożliwia wykrycie roztoczy \textit{Varroa} na pszczołach, a~rozwiązanie Dunkera wraz z~zespołem \cite{Dunker2021} pozwala na ocenę ilości pyłku transportowanego przez każdą z~pszczół.

Podejściem, które, mimo swej prostoty, wciąż nie zostało szeroko przyjęte jest zastosowanie do wykrywania pszczół czujników pojemnościowych.
Wykorzystanie różnicy pojemności elektrycznej kondensatora wywołanej ruchem pszczoły zostało pierwszy raz zastosowane w~pracy Campbell wraz z~zespołem \cite{Campbell2005}, która zaprojektowała czujnik składający się z~pierścieniowych okładek, między którymi poruszają się owady.
Zmodyfikowana wersja podobnego systemu została zaprezentowana przez Perraulta i~Teachmana \cite{Perrault2016}, którzy zwrócili uwagę na szereg zalet tego rozwiązania: czujnik pojemnościowy nie jest podatny na zabrudzenia i~nie wymaga częstego czyszczenia; ponadto, jest całkowicie nieszkodliwy dla pszczół.
Zwrócono uwagę na podatność systemu na błąd wynikający z~grupowania się pszczół.
Problem ten zaadresował Bermig wraz z~zespołem \cite{Bermig2020}, konstruując urządzenie \textit{BeeCheck} posiadające siedem bramek pojemnościowych w~torze ruchu pszczół.
W ostatnim czasie popularność pojemnościowych liczników pszczół zaczęła rosnąć, pojawił się nawet internetowy artykuł instruktażowy opisujący, jak wykonać podobne urządzenie samodzielnie \cite{mathematastic}.



\subsection{Cel i~zakres pracy}
Celem niniejszej pracy jest stworzenie urządzenia zliczającego pszczoły w~ulu opartego o~zestaw czujników pojemnościowych.
Ma ono być montowane na wylotku (szczelinie, przez którą pszczoły wchodzą do ula i~opuszczają go) tak, by cały ruch owadów odbywał się poprzez czujniki.
Jego zadaniem jest wykrycie każdej pojawiającej się pszczoły, określenie chwili jej przejścia, a~także kierunku jej ruchu (wejście/wyjście) -- w~celu umożliwienia zliczania pszczół aktualnie znajdujących się w~ulu.

Na pracę składać się będą rozwiązania dwóch głównych problemów:
\begin{enumerate}
    \item zaprojektowanie czujników pojemnościowych i~metody akwizycji ich sygnałów, a~także konstrukcja urządzenia wyposażonego w~ich szereg,
    \item opracowanie algorytmu detekcji, umożliwiającego wykrywanie pszczół na podstawie wyjść poszczególnych czujników pojemnościowych, a~także jego implementacja na platformie sprzętowej urządzenia.
\end{enumerate}
Tworzony system zostanie kompleksowo przetestowany, zarówno w~warunkach laboratoryjnych, jak i~w pracy w~środowisku rzeczywistym, by zweryfikować skuteczność zliczania pszczół i~sprawdzić jego podatność na błędy.

\clearpage
\section{Budowa urządzenia}
\subsection{Założenia projektu}
Aby możliwe było wykorzystanie urządzenia zliczającego pszczoły do diagnostyki ula, musi ono spełniać szereg kryteriów.
Po pierwsze, naturalne funkcjonowanie kolonii nie może zostać w znaczącym stopniu zakłócone. Należy zadbać, by ruch pszczół został zaburzony w minimalnym stopniu, a~wentylacja gniazda nie była zbytnio ograniczona.
Ponadto, istotna jest odporność systemu na warunki atmosferyczne i~aktywność pszczół: m.in. na zmiany nasłonecznienia, wilgotności, zbieranie się pyłku.

Urządzenie ma dostarczać informację o chwilach wejścia/wyjścia, a także kierunku poruszania się pszczoły.
Wyprodukowanie czujnika nie może mimo to wymagać trudno dostępnych lub drogich komponentów, aby możliwe było intensywne iteracyjne prototypowanie i~ewentualne szerokie wykorzystanie stworzonego systemu.

\subsection{Prototypowy czujnik pojemnościowy}
By spełnić wszystkie wymienione w poprzedniej sekcji założenia, stworzono autorski układ wykrywający pszczoły oparty o~czujnik pojemnościowy.
  \subsubsection{Zasada działania}
Pojemność elektryczna kondensatora zależy od przenikalności elektrycznej ośrodka między jego okładkami. W przypadku kondensatora płaskiego, pojemność opisywana jest wzorem:
\[
C = \frac{\varepsilon_r\,\varepsilon_0\,S}{d}
\]
gdzie $C$~--~pojemność kondensatora, $\varepsilon_r$~--~względna przenikalność elektryczna ośrodka, $\varepsilon_0$~--~przenikalność elektryczna próżni, $S$~--~pole powierzchni okładki, $d$~--~odległość między okładkami \cite{efizyka-kondensator}.

Przenikalność elektryczna powietrza wynosi $\varepsilon_r = 1.00$ \cite{Hector1936}. Pszczoła nie jest jednorodnym ośrodkiem i jej parametry elektryczne nie były nigdy mierzone, jednak w znacznej części składa się ona z wody – substancji o wysokim $\varepsilon_r = 80.2$ (przy $20^\circ$C) \cite{Archer1990}.
W literaturze proponowana jest arbitralna wartość przenikalności elektrycznej pszczoły równa $\varepsilon_r = 50$, która znajduje pośrednie potwierdzenie w eksperymentach \cite{Campbell2005}.

Wprowadzenie pszczoły pomiędzy okładki kondensatora spowoduje zatem wzrost średniej przenikalności elektrycznej ośrodka, zwiększając tym samym pojemność układu. Na rysunku \ref{fig:bee-in-capacitor} przedstawiony został układ kondensatora oraz teoretyczny wykres pojemności elektrycznej w zależności od położenia przemieszczającej się pszczoły. Na podstawie kształtu przebiegu pojemności określić można momenty wejścia i opuszczenia kondensatora (odpowiednio $t_{\mathrm{enter}}$ i $t_{\mathrm{leave}}$ na rysunku).

\begin{figure}
    \centering
    \includegraphics{tex//img/bee-in-capacitor.pdf}
    \caption{Schemat układu kondensatora z poruszającą się ruchem jednostajnym pszczołą oraz odpowiadający przebieg pojemności elektrycznej $C_1$.}
    \label{fig:bee-in-capacitor}
\end{figure}

Przedstawiony układ nie umożliwia natomiast określenia kierunku poruszania się pszczoły w czujniku. Aby zapewnić tę funkcję układ rozszerzony jest o drugi kondensator. Schemat takiego układu został przedstawiony na rysunku \ref{fig:bee-in-c2}. Wyjście czujnika stanowi sygnał $C_1 - C_2$. Założono, że kondensatory są identyczne, a więc gdy nie znajduje się w nich pszczoła, to $C_1 - C_2 = 0$. Ruch pszczoły między okładkami generuje na wyjściu bipolarny, środkowosymetryczny impuls, który  daje się zidentyfikować na podstawie kształtu, pozwalając na wykrycie pszczół w czujniku, oraz pozwala określić kierunek ruchu pszczoły na podstawie kolejności występowania piku dodatniego i ujemnego. Na rysunku \ref{fig:bee-in-minus-c2} przedstawiono sytuację, w której pszczoła porusza się w odwrotnym kierunku.

\begin{figure}
    \centering
    \includegraphics{tex//img/bee-in-c2.pdf}
    \caption{Schemat i przebieg $C_1 - C_2$ dla układu dwóch kondensatorów.}
    \label{fig:bee-in-c2}
\end{figure}

\begin{figure}
    \centering
    \includegraphics{tex//img/bee-in-minus-c2.pdf}
    \caption{Schemat i przebieg $C_1 - C_2$ dla układu dwóch kondensatorów -- pszczoła porusza się w odwrotnym kierunku.}
    \label{fig:bee-in-minus-c2}
\end{figure}

Przedstawione ilustracje mają na celu wyłącznie prezentację podstawowej zasady działania czujnika pojemnościowego, a podczas ich tworzenia pominięte zostało wiele zjawisk fizycznych.
W praktyce, kondensatory o płaskich okładkach nie sprawdzają się takim zastosowaniu -- są podatne na zewnętrzne zakłócenia, a także muszą być znacznie oddalone od siebie by uniknąć wzajemnego zakłócania \cite{Campbell2005}.
Mimo to, rozwiązania znalazły wykorzystanie w pracach naukowych, na przykład urządzenie \textit{BeeCheck} \cite{Bermig2020}, kompensujące wady płaskich okładek wykorzystaniem 7 kondensatorów w czujniku.

Znacznie skuteczniejszym podejściem jest wykorzystanie okładek pierścieniowych. Na rysunku \ref{fig:tunnel-concept} przedstawiona została ogólna idea tego rozwiązania. Na czujnik składa się rurka z izolatora, przez którą przechodzi owad, oraz trzy pierścienie z przewodnika.
Konstrukcja ta realizuje taki sam układ elektryczny jak analizowano poprzednio, przy czym kondensatory $C_1$ i $C_2$ mają wspólną okładkę (środkowy pierścień – wyprowadzenie $C$). Kondensator $C_1$ znajduje się między wyprowadzeniami $L$ i $C$, natomiast kondensator $C_2$ między $R$ i $C$.
Ułożenie to, pozwalając na równoczesne ładowanie obu kondensatorów, umożliwi pomiar różnicy pojemności $C_1 - C_2$.

\begin{figure}
    \centering
    \includegraphics{tex//img/tunnel-concept.pdf}
    \caption{Konstrukcja czujnika opartego na elektrodach pierścieniowych.}
    \label{fig:tunnel-concept}
\end{figure}

Działanie czujnika tego typu zostało zasymulowane numerycznie przez Campbell wraz z zespołem \cite{Campbell2005}. Przyjęto: średnicę wewnętrzną tunelu 15 mm, okładki kondensatora wykonane z drutu o średnicy 1.7 mm, odległość między okładkami 7 mm. Pszczołę zamodelowano jako cylinder o długości 12 mm i średnicy 6mm. Wyniki tej symulacji przedstawione zostały na rysunkach \ref{fig:campbell-symulacja-model} i \ref{fig:campbell-symulacja-wynik}.
Model czujnika zakładał dodatkowe obudowanie całości stalowym ekranem i zastosowanie polistyrenowych zaślepek. Na rysunku \ref{fig:campbell-symulacja-model} widać, że pszczoła (symulowana jako cylinder o~$\varepsilon_r=50$) wchodząca do tunelu powoduje niesymetryczne rozłożenie pól elektrycznych w kondensatorach.
Przejście pszczoły skutkuje pojawieniem się na wyjściu czujnika asymetrycznych, bipolarnych impulsów (rysunek \ref{fig:campbell-symulacja-wynik}), podobnych do tych na przebiegach z rozważań teoretycznych (rysunki \ref{fig:bee-in-c2}, \ref{fig:bee-in-minus-c2}). W symulacji przetestowano dodatkowo odpowiedź układu na pszczoły różnych rozmiarów (impulsy $a$, $b$ na rysunku \ref{fig:campbell-symulacja-wynik}). Zaobserwować można pewną różnicę w wyglądzie wygenerowanych przebiegów, jednak ogólny kształt nadal pozwala na rozpoznanie momentu oraz kierunku przejścia pszczoły.

\begin{figure}
    \centering
    \includegraphics{tex//img/campbell-symulacja-model.pdf}
    \caption{Przekrój poprzeczny czujnika oraz linie ekwipotencjalne -- symulacja przeprowadzona przez Campbell wraz z zespołem. \cite{Campbell2005}}
    \label{fig:campbell-symulacja-model}
\end{figure}

\begin{figure}
    \centering
    \includegraphics{tex//img/campbell-symulacja-wynik.pdf}
    \caption{Impulsy wygenerowane przez pszczołę przechodzącą przez czujnik. Wyniki symulacji. Różnica pojemności kondensatorów była przetwarzana na napięcie przez liniowy przetwornik. (a)~--~duża pszczoła opuszczająca ul; (b)~--~mała pszczoła wchodząca do ula. \cite{Campbell2005}}
    \label{fig:campbell-symulacja-wynik}
\end{figure}

  \subsubsection{Układ pomiarowy}
Sygnał wyjściowy omawianego czujnika pojemnościowego jest trudny do zmierzenia. Pojemności kondensatorów $C_1$, $C_2$ wynoszą zaledwie ok. 0,5 pF \cite{Campbell2005}, a przy pojawieniu się pszczoły zwiększają się tylko o kilka--kilkanaście procent.
Pomiar tak małych pojemności wymaga specjalistycznego sprzętu oraz niezwykłej dbałości podczas projektowania układu.
Pojemności pasożytnicze ścieżek na płytce drukowanej mogą być przy słabym projekcie nawet o rząd wielkości wyższe niż pojemność badanego kondensatora.
Szczegółowe założenia, którymi kierowano się przy projekcie PCB by zminimalizować ich wpływ, zostały szczegółowo opisane w sekcji \ref{pcb}.

W literaturze znaleźć można przykładowe układy pomiarowe, które działają na zasadzie różnicy przesunięcia fazowego. Wspólna okładka kondensatorów ekscytowana jest sygnałem okresowym (generowanym z pomocą układu Intersil ICL8038 \cite{Campbell2005} lub NE555P \cite{mathematastic}). Kondensatory czujnika połączone są z masą układu przez rezystory, tworząc razem filtr górnoprzepustowy przesuwający fazę sygnału w zależności od pojemności. Przesunięte sygnały są następnie odejmowane z pomocą precyzyjnego wzmacniacza różnicowego.
Następnie następuje demodulacja i wzmocnienie sygnału. W efekcie, wyjście układu stanowi sygnał o napięciu proporcjonalnym do różnicy pojemności obu kondensatorów. 

Taki sposób pomiaru został z powodzeniem zastosowany w urządzeniach zliczających pszczoły, jednak jest bardzo kosztowny. Konieczne jest użycie układów scalonych o wysokiej precyzji – szacunkowy koszt układów scalonych potrzebnych do obsłużenia jednej pary kondensatorów to 231 zł\footnote{Ceny na dzień 09.10.2025 za: ICL8038, AD620, AD630, LF411.}. Całe urządzenie ma się składać z ośmiu równoległych tuneli, łączna cena układów scalonych wyniosłaby 1848 zł znacznie przekraczając limit kosztu całego urządzenia.

Alternatywnym podejściem jest zastosowanie dedykowanego scalonego układu przetwornika pojemnościowo-napięciowego.
Przykładem takiego układu odpowiedniego do małych pojemności jest AD7746, zastosowany w pracy Paula Perraulta \cite{Perrault2016}.
Jest on przetwornikiem dwukanałowym, wystarczy zatem jeden na oba kondensatory w pojedynczym czujniku \cite{ad7746}. Koszt zakupu jednego układu scalonego to 67 zł (sumarycznie 536 zł na 8~tuneli), co jednak wciąż przekracza zakładany budżet.\newline

W ramach niniejszej pracy opracowany został prosty i niezwykle tani sposób akwizycji sygnału z czujnika pojemnościowego. Jakość pomiaru jest znacznie niższa niż w opisanych wcześniej rozwiązaniach, lecz uzyskane przebiegi czasowe pozwalają skutecznie wykrywać ruch pszczół i określać jego kierunek.
Schemat proponowanego układu przedstawiony jest na rysunku \ref{fig:uklad-pomiarowy}.
Jego działanie opiera się na różnicy czasu ładowania kondensatorów o różnych pojemnościach.

\begin{figure}
    \centering
    \includegraphics{tex//img/ladowanie-kondensatora.pdf}
    \caption{Ładowanie kondensatora: schemat układu i przebiegi napięć.}
    \label{fig:ladowanie kondensatora}
\end{figure}


Rozważmy układ pojedynczego kondensatora (rysunek \ref{fig:ladowanie kondensatora}). Odpowiedź $u_C(t)$ na skok napięcia $u$ od 0 do $U$ dana jest jako:
\[
u_C(t) = U\,\left(1-e^{-\frac{t}{\tau}}\right)
\]

gdzie $\tau = R\,C$ --- stała czasowa układu \cite{ladowanie-kondensatora}. Przykładowe przebiegi odpowiedzi skokowej przedstawione zostały na rysunku \ref{fig:ladowanie kondensatora}. Większa stała czasowa wiąże się z wolniejszym narastaniem napięcia na kondensatorze. 
Przyjmując pewne napięcie progowe $U_{\mathrm{prog}}$, możemy określić czas ładowania kondensatora jako czas od skoku $u$ do przekroczenia $U_{\mathrm{prog}}$ przez $u_C$.
Ponieważ $u_C(t)$ jest funkcją rosnącą, to dla stałych czasowych $\tau_1<\tau_2$ odpowiadające im czasy ładowania spełniają zależność $t_1 < t_2$.
Odpowiedni dobór $U_\mathrm{prog}$ ma wpływ na jakość działania systemu: przy niskim progu $t_1$ oraz $t_2$ będą niewielkie i bardzo zbliżone do siebie. Wysoki próg zwiększa podatność na szumy w pomiarze $u_C$. Wybór konkretnej wartości $U_\mathrm{prog}$ został szerzej opisany dalej.

W układzie czujnika każdy z kondensatorów ładowany jest przez taką samą rezystancję –-stałe czasowe podukładów zależą wyłącznie od pojemności elektrycznej samych kondensatorów. Metoda wykrywania pszczół stworzona w ramach niniejszej pracy nie została zatem oparta bezpośrednio na różnicy pojemności kondensatorów, tylko na różnicy w czasie ich ładowania.

Wyzwaniem technicznym jest sam pomiar czasu ładowania kondensatora o tak niewielkiej pojemności. Przyjmując, jak wcześniej, orientacyjną pojemność kondensatora równą 0,5 pF, wyznaczyć można szacowaną stałą czasową układu (z rezystorem 1M$\Omega$):
\[
\tau = R\cdot C = 1\mathrm{M}\Omega\cdot0.5\mathrm{pF}
 =  
10^6\Omega\cdot0.5\cdot10^{-12}\mathrm{F}
 = 
2\cdot10^{-6}\mathrm{s}
 = 
2 \mu\mathrm{s}
\]
\begin{figure}
    \centering
    \includegraphics{tex//img/uklad-pomiarowy.pdf}
    \caption{Schemat układu pomiarowego.}
    \label{fig:uklad-pomiarowy}
\end{figure}
Jak widać, stała czasowa układu przyjmuje wartości rzędu pojedynczych mikrosekund. Precyzyjny pomiar takich czasów jest zasadniczo niemożliwy z wykorzystaniem sprzętu o niskim koszcie.
Z tego powodu, w niniejszej pracy zastosowana została pewna sztuczka: ładowanie kondensatora mierzone jest nie w jednostkach czasu, ale w liczbie wykonań pojedynczych instrukcji mikrokontrolera monitorującego stan wyjść czujnika. Działanie to formalnie nie daje faktycznego pomiaru czasu, jednak wynik pozwala się jako taki pomiar zastosować – co zostanie udowodnione w dalszych częściach pracy.
Na rysunku \ref{fig:uklad-pomiarowy} przedstawione zostało połączenie czujnika z mikrokontrolerem, natomiast pseudokod \ref{alg:akwizycja} pokazuje działanie algorytmu akwizycji danych.

\begin{algorithm}
\caption{Algorytm akwizycji pomiaru}
\label{alg:akwizycja}
\begin{algorithmic}
\State $n_L \gets 0$\
\State $n_R \gets 0$\
\State $s_L \gets 0$\
\State $s_R \gets 0$\
\State \texttt{write\_gpio(}Charge\texttt{, 1)}\
\While{$s_L = 0$ \textbf{or} $s_R = 0$ }
    \State $s_L \gets $\texttt{read\_gpio(}$L$\texttt{)}\
    \State $s_R \gets $\texttt{read\_gpio(}$R$\texttt{)}\
    \If{$s_L$ = 0}
        \State $n_L \gets n_L + 1$
    \EndIf
    \If{$s_R$ = 0}
        \State $n_R \gets n_R + 1$
    \EndIf
\EndWhile
\State \texttt{write\_gpio(}Charge\texttt{, 0)}\
\State \texttt{return} $n_L - n_R$
\end{algorithmic}
\end{algorithm}

Algorytm inicjalizuje dwa liczniki iteracji odpowiadające dwóm kondensatorom: $n_l$ oraz $n_r$. Następnie na pin \textit{Charge} wystawiany jest stan wysoki, co rozpoczyna ładowanie kondensatorów.
Program iteracyjnie inkrementuje liczniki dopóki piny GPIO $L$ oraz $R$ nie zostaną podniesione do stanu wysokiego.
Stan wysoki występuje po przekroczeniu przez napięcie kondensatora progu jedynki logicznej mikrokontrolera – stanowi on w tym przypadku $U_\mathrm{prog}$ układu.
Wartości progowe są dobierane tak, by minimalizować czas odpowiedzi pinu, ale również podatność na szumy. Optymalizowane przez projektantów mikrokontrolerów kryteria są podobne do tych, które musi spełniać odpowiednio dobrany próg $U_\mathrm{prog}$, można się zatem spodziewać, że system wykorzystujący cyfrowe piny GPIO będzie działał prawidłowo.
Po naładowaniu kondensatorów i ustaleniu $n_L$ oraz $n_R$ pin \textit{Charge} ponownie ustawiany jest do stanu niskiego. Przed wykonaniem kolejnego pomiaru kondensatory muszą zostać rozładowane, nie wymaga to jednak rozbudowywania układu, ponieważ zastosowany rodzaj kondensatora szybko rozładowuje się samoistnie.
Należy jednak pamiętać, aby na kolejnych etapach projektu wprowadzić ograniczenie częstotliwości próbkowania, aby dać ładunkom elektrycznym czas na rozproszenie.

Zaprezentowany algorytm akwizycji oferuje rozdzielczość pomiaru zależną od czasu wykonania instrukcji znajdujących się wewnątrz pętli \textit{while}.
Najdłuższą operacją jest odczyt stanu pinu (\texttt{read\_gpio}) – dla zastosowanego mikrokontrolera ESP32-C3 może on trwać $0.05 - 0.17\,\mu\mathrm{s}$ (w zależności od implementacji) \cite{ESP32C3Datasheet}\cite{microcontroller-benchmarks}. Nawet przy konieczności dwukrotnego wykonania instrukcji, rozdzielczość taka jest wystarczająca na potrzeby tworzonego urządzenia.

Alternatywnym podejściem, niewykorzystanym w niniejszej pracy z braku wyraźnej potrzeby, jednak zapewniającym znacznie wyższą rozdzielczość byłoby wykorzystanie zewnętrznych przerwań mikrokontrolera. Modyfikacja przedstawionego algorytmu polega na usunięciu odczytu GPIO z pętli \textit{while}, zamiast czego $s_L$ oraz $s_R$ ustawiane są na wartość 1 w ramach obsługi przerwań EXTI wynikających z wykrycia zboczy narastających na pinach $L$ i $R$.

  
  \subsubsection{Realizacja sprzętowa}


\begin{figure}
    \centering
    \includegraphics[width=0.8\textwidth]{tex/img/prototyp.png}
    \caption{Prototypowy czujnik pojemnościowy.}
    \label{fig:prototyp}
\end{figure}

W celu przetestowania działania opisanego systemu zbudowana została prototypowa konstrukcja czujnika (rysunek \ref{fig:prototyp}).
Tunel czujnika wykonany został z rurki z przezroczystego tworzywa sztucznego, natomiast okładki wykonano z obręczy wygiętych z drutu miedzianego i zlutowanych w miejscu łączenia.
Wyprowadzenia okładek kondensatora zostały wykonane z izolowanych przewodów o długości ok. 15 cm.
W tabeli \ref{tab:prototyp} zawarte zostały parametry opracowanej konstrukcji.
Obrane wymiary oparte zostały na wynikach optymalizacji podobnego czujnika w pracy Campbell wraz z zespołem, w której przeprowadzono symulację elektrostatyczną, by następnie dobrać średnice $s$ i rozstaw okładek $d$, tak by maksymalizować amplitudę sygnału generowanego przez pszczołę w czujniku \cite{Campbell2005}.
Wypracowane w tej pracy kryteria to:
\begin{enumerate}
    \item \(\frac{1}{2}\,s+t < d <s + 2\,t\) -- rozstaw okładek powinien być większy od promienia okładki, ale nie większy od średnicy;
    \item $g$ -- średnica drutu, z którego wykonane są okładki powinna być jak największa -- na ile pozwolą inne ograniczenia konstrukcyjne.
\end{enumerate}
\begin{figure}
    \centering
    \includegraphics{tex/img/prototyp-tech.pdf}
    \vspace{2em}
    \centering
    \begin{tblr}{cl|r}
        \textbf{Oznaczenie} & \textbf{Opis} & \textbf{Wartość} [mm] \\
        \hline
       $\textbf{\textit{l}}$ & Długość tunelu & 120  \\
       $\textbf{\textit{D}}$ & Odległość do 1. okładki & 8 \\
       $\textbf{\textit{d}}$ & Rozstaw okładek & 9 \\
       $\textbf{\textit{g}}$ & Średnica drutu okładki & 1.5 \\
       $\textbf{\textit{t}}$ & Grubość ścianki tunelu & 0.85 \\
       $\textbf{\textit{s}}$ & Średnica wewn. tunelu & 14.3 \\
    \end{tblr}
    \captionof{table}{Wymiary prototypowego tunelu}
    \label{tab:prototyp}
\end{figure}

W ramach niniejszej pracy założono, że wewnętrzna średnica tunelu musi wynosić około 14 mm, aby zapewnić swobodny ruch pszczół, a przy tym uniemożliwić równoległe poruszanie się kilku osobników obok siebie.
Średnica zewnętrzna zastosowanej rurki z tworzywa sztucznego to 16 mm, co daje promień okładki wynoszący 8 mm. Dla spełnienia wyżej wymienionego kryterium obrano rozstaw okładek $d=9$ mm.
Do wykonania okładek wybrano drut miedziany o średnicy 1.5 mm.

\begin{figure}
    \centering
    \includegraphics{tex/img/uklad-prototyp.pdf}
    \caption{Schemat układu elektronicznego prototypowego tunelu.}
    \label{fig:uklad-prototyp}
\end{figure}


Na uniwersalnej płytce prototypowej stworzono układ elektroniczny zgodny tym przedstawionym na rysunku \ref{fig:uklad-pomiarowy}. Na potrzeby przetestowania prototypu pominięto jeden z kondensatorów w tunelu i do mikrokontrolera podłączono jedynie okładkę lewego kondensatora. Zastosowany został wspominany wcześniej mikrokontroler ESP32-C3 na płytce deweloperskiej ESP32-C3 Super Mini \cite{ESP32C3Datasheet}. Moduł ten wyposażony jest w konwerter USB-UART umożliwiający komunikację urządzenia z komputerem z wykorzystaniem złącza USB. Na mikrokontroler zaimmplementowana została uproszczona (mierząca tylko jedną z bramek) wersja algorytmu \ref{alg:akwizycja}:


\inputminted[label={prototype/main.cpp}, linenos, frame=single, framesep=2em]{cpp}{code/prototype-main.cpp}



  \subsubsection{Testy}
  
\subsection{Układ elektroniczny}
  \subsubsection{Dobór komponentów}
  \subsubsection{Realizacja płytki PCB} \label{pcb}
\subsection{Oprogramowanie mikrokontrolerów}
  \subsubsection{Algorytm akwizycji}
  \subsubsection{Komunikacja między mikrokontrolerami}
  \subsubsection{Komunikacja ze światem zewnętrznym}
\subsection{Testy}
\subsection{Modyfikacja układu}
\subsection{Realizacja obudowy}
\clearpage
\section{Algorytm detekcji} \label{chapter:algorytm}
\subsection{Zbiór danych przykładowych}
  \subsubsection{Testy urządzenia w pasiece}
  \subsubsection{Ręczna anotacja danych}
\subsection{Wstępne przetwarzanie danych}
  \subsubsection{Filtr uśredniający}
  \subsubsection{Usuwanie trendu}
\subsection{Opis proponowanych algorytmów}
  \subsubsection{Automat stanowy}
  \subsubsection{Korelacja wzajemna z wzorcem}
  \subsubsection{Klasyfikator na cechach sygnału}
\subsection{Wyniki testów}
\subsection{Wybór algorytmu}
\subsection{Implementacja algorytmu na sprzęcie}
\clearpage
\section{Test integracyjny}

Na tym etapie realizacji pracy możliwe było przetestowanie pełnej funkcjonalności stworzonego urządzenia w finalnym teście integracyjnym.
Ostatnim, co pozostało do wykonania przed jego przeprowadzeniem jest implementacja wybranego algorytmu detekcji pszczół na platformie sprzętowej ESP32-C3.

\subsection{Implementacja algorytmu na sprzęcie} \label{sekcja:implementacja}

Ograniczenia narzucane przez wykorzystywany mikrokontroler wymagają przeanalizowania etapów algorytmu pod kątem złożoności obliczeniowej i rozważenia optymalizacji implementacji niektórych z nich.
Przytoczone poniżej wartości w notacji \textit{big O} zostały wyznaczone dla wykonania danej procedury w pojedynczej pętli urządzenia -- czyli dla jednej przychodzącej próbki sygnału wejściowego.
Ponadto, założona została implementacja bezpośrednio wynikająca z przedstawionych wcześniej wzorów matematycznych, bez dodatkowych optymalizacji.

Automat stanowy ma złożoność obliczeniową $O(1)$. Jest niezwykle prosty i nadaje się świetnie do pracy w ograniczonych zasobach.
Problematycznym krokiem algorytmu jest natomiast progowanie adaptacyjne.
Opiera się ono na wyznaczaniu pewnego kwantylu z buforu ostatnich próbek -- wymaga zatem posortowania próbek z całego wymaganego okna.
Z tego względu, jego złożoność obliczeniowa to $O(n\log{n})$.
Analogicznie w przypadku kroku usuwania trendu: korzysta on z mediany próbek w buforze o innej długości -- konieczne jest kolejne sortowanie ($O(n\log{n})$).
Dodatkowo, filtr średniej kroczącej, w podstawowej implementacji, posiada złożoność obliczeniową $O(n)$, ponieważ w każdym kroku uśrednia próbki ze swojego buforu.

Występowanie w implementacji etapów o złożoności $O(n\log{n})$ samo w sobie nie stanowi problemu, dopóki algorytm wykonuje się wystarczająco szybko.
Aby sprawdzić, czy konieczna jest jego optymalizacja, przeprowadzony został test czasu wykonania pojedynczej operacji usunięcia trendu z próbki. Założono okno $K=1500$, zgodnie z najlepszą uzyskaną wcześniej konfiguracją.
Algorytm uruchomiony został wiele razy, a zmierzony czas uśredniono. Wyniki dla komputera stacjonarnego oraz ESP32-C3 przedstawione zostały w tabeli \ref{tab:czas-wykonania}.
Wyznaczony czas wykonania na mikrokontrolerze (3.79 milisekundy) jest znacząco za wysoki: pomnożony przez 8 tuneli uniemożliwiłby, sam w sobie, pracę urządzenia z zakładaną częstotliwością próbkowania 100 Hz.
Wynika z tego, że konieczne jest zoptymalizowanie implementacji metody usuwania trendu.

\begin{table}[htb]
\centering
\caption{Porównanie czasów wykonania implementacji usuwania trendu.}
\label{tab:czas-wykonania}
\begin{tblr}{
  cells = {c},
  column{1} = {r},
  vline{2} = {-}{},
  hline{2} = {-}{},
}
\textbf{Wersja implementacji}        & \textbf{Komputer stacjonarny} & \textbf{ESP32-C3} \\
bez optymalizacji    & 51.63 $\mu s$                        & 3783 $\mu s$            \\
zoptymalizowana & 6.075 $\mu s$                       & 19.65 $\mu s$             
\end{tblr}
\end{table}


Proponowana optymalizacja opiera się na spostrzeżeniu, że w buforze zawsze dodawana i usuwana jest dokładnie jedna próbka -- pełne sortowanie całości jest zatem nadmiarowe.
Do reprezentacji buforu próbek wykorzystane zostanie zbalansowane binarne drzewo poszukiwań (BST).
Jest to struktura danych, która przechowuje posortowane elementy, umożliwiając ich dodawanie i usuwanie ze złożonością $O(\log n)$ \cite{cormen2019}.
Dostęp do środkowego elementu (mediany) odbywa się w czasie $O(n)$.
W każdym wykonaniu pętli programu wykonane zostanie zatem odnalezienie ($O(\log n)$ i usunięcie z drzewa pojedynczego elementu (najstarszej próbki), dodanie najnowszej próbki, oraz pobranie mediany.
Sumarycznie, rozwiązanie takie pozwoli na usuwanie trendu z próbek z całkowitą złożonością obliczeniową $O(n)$, co powinno dać znaczną poprawę czasu wykonania względem wersji bez optymalizacji.
W języku C++ zbalansowane binarne drzewo poszukiwań jest dostępne jako element biblioteki standardowej, pod nazwą \texttt{std::multiset} \cite{cppref_multiset}.
Opracowana implementacja została przedstawiona na przedstawionym poniżej fragmencie kodu:

\vspace{1em}
\vbox{
\begin{minted}
[
    label={include/detection.h > RollingMedianDetrender}, 
    linenos, 
    frame=single, 
    framesep=0.5em, 
    fontsize=\footnotesize
]{cpp}
class RollingMedianDetrender {
  public:
    RollingMedianDetrender() {
        // Na początku, wypełnia BST próbkami z buforu
        for (size_t i = 0; i < kDetrendWindow; ++i) {
            float value = x.at(i);
            bst.insert(value);
        }
    }

    float update(float newValue) {
        float oldestValue = x.at(0);       // Aktualizacja buforu
        x.push_back(newValue);

        bst.erase(bst.find(oldestValue));  // Aktualizacja BST
        bst.insert(newValue);

        float median = *next(bst.rbegin(), kDetrendWindow/2); // Pobranie mediany
        return newValue - median;
    }

  private:
    CircularBuffer<kDetrendWindow> x;
    std::multiset<float> bst;
};
\end{minted}
}
\vspace{1em}

Warto zwrócić uwagę, że oprócz samego drzewa binarnego, przechowywany jest również oryginalny bufor próbek.
Jest on konieczny do zachowania informacji o kolejności próbek -- należy wiedzieć, jaka jest najstarsza próbka w buforze, by usunąć ją z drzewa.
Powstaje dodatkowy narzut pamięciowy, ale ESP32-C3 posiada wystarczająco dużo adresów RAM, by nie stanowiło to problemu.
Bufor reprezentowany jest przez strukturę \texttt{CircularBuffer}, zaimplementowaną na potrzeby niniejszej pracy.
Jej celem jest przechowywanie buforu ostatnich próbek w postaci, która nie wymaga przesuwania całej zawartości jego pamięci o jedno pole w momencie pojawiania się nowej próbki; zamiast tego przesuwany jest wskaźnik na najnowszy element.
Implementacja klasy \texttt{CircularBuffer} została przedstawiona poniżej.

\vspace{1em}
\vbox{
\begin{minted}
[
    label={include/detection.h > CircularBuffer}, 
    linenos, 
    frame=single, 
    framesep=0.5em, 
    fontsize=\footnotesize
]{cpp}
template<size_t N>
class CircularBuffer {
  public:
    CircularBuffer() : m_head(0) {
        for (size_t i = 0; i < N; ++i) {
            m_buffer[i] = 0.0f;
        }
    }

    void push_back(float value) {
        m_buffer[m_head] = value;
        m_head = (m_head + 1) % N;
    }

    float at(size_t index) const {
        size_t physicalIndex = (m_head + index) % N;
        return m_buffer[physicalIndex];
    }
    
    size_t size() const { return N; }

  private:
    float m_buffer[N];
    size_t m_head;
};
\end{minted}
}
\vspace{1em}


Wyniki z tabeli \ref{tab:czas-wykonania} pozwalają jednoznacznie potwierdzić, że optymalizacja przyspieszyła działanie algorytmu.
Obliczenia zajęły w tym przypadku zaledwie 19.65 $\mu s$ -- wystarczająco mało, by system działał z prawidłowym czasem próbkowania.

Kolejnym etapem algorytmu o złożoności $O(n\log n)$, wymagającym tym samym optymalizacji, jest obliczanie wartości dynamicznego progu $\gamma(k)$.
Wąskim gardłem procesu jest wyznaczanie wartości $\hat{Q}_y(q,\,K,\,k)$ ($q$-tego kwantylu z buforu dłufości $K$).
Zastosowano analogiczną optymalizację jak w przypadku mediany -- jedyną różnicą jest pozycja wartości pobieranej z BST.
W tym przypadku, zamiast na pozycji środkowej, interesująca próbka leży na pozycji $q\cdot K$.

Dalsza optymalizacja implementacji algorytmu detekcji, jednak dla elegancji rozwiązania zmodyfikowano kod filtra średniej kroczącej w sposób analogiczny do przedstawionych uprzednio usprawnień.
Zamiast w każdej pętli programu uśredniać całe okno na nowo, całkowita wartość średniej jest jedynie aktualizowana o różnice między usuwaną, najstarszą próbką, a dodawaną -- najnowszą.
W ten sposób z pracy w czasie $O(n)$ usprawniono ten krok algorytmu do $O(1)$.

Łączny czas wykonania całego algorytmu detekcji na mikrokontrolerze ESP32-C3 wynosi 245 mikrosekund.
Dla wszystkich ośmiu tuneli urządzenia daje to sumaryczny czas nie przekraczający 2 milisekund, wykorzystywane jest zatem jedynie niecałe 20\% dostępnego czasu przy próbkowaniu 100 Hz.
Dalsza optymalizacja nie jest konieczna -- choć byłaby możliwa poprzez zastosowanie do reprezentacji buforu kilku drzew binarnych, balansowanych tak, by na początku jednego z nich znajdowała się zawsze interesująca próbka, np. mediana.


\subsection{Test zliczania pszczół}
Stworzony program zawierający opracowany algorytm detekcji pszczół wgrano na skonstruowane urządzenie.
W celu weryfikacji działania systemu powtórnie przeprowadzono eksperyment z kawałkiem wilgotnej gąbki na żyłce symulującym pszczołę.
Przez każdy z tuneli urządzenia przeciągnięto go czterokrotnie: dwa razy w każdym kierunku.
Urządzenie połączone przez USB z komputerem stacjonarnym stale raportowało stany liczników pszczół $bee_i$, odpowiadające poszczególnym tunelom $i=0\dots7$.
Uzyskane przebiegi czasowe zostały przedstawione na rysunku \ref{fig:test-integracyjny}.
\begin{figure}[htb]
    \centering
    \includegraphics{tex/img/integration-test.pdf}
    \caption{Przebiegi liczników $bee_i$, $i=1\dots7$, zebrane podczas testu integracyjnego.}
    \label{fig:test-integracyjny}
\end{figure}
Na każdym z przedstawionych wykresów widać po dwa fragmentu, na których licznik przechodzi z zera na jedynkę.
Skoki wartości $bee_i$ odpowiadają momentom przeciągania przez czujnik obiektu, i występują w prawidłowej kolejności.
Można jednoznacznie wnioskować, że algorytm detekcji zadziałał prawidłowo w teście integracyjnym:
wygenerował on prawidłowe momenty i kierunki wykrywanych zdarzeń.
Na tym etapie pracy, stworzony system jest w pełni sprawny -- można przejść do uruchamiania go w warunkach rzeczywistych.
\clearpage
\section{Rezultaty pracy w~warunkach rzeczywistych}

Stworzony system zliczania pszczół ponownie uruchomiono w~jednej z~odwiedzonych wcześniej pasiek.
Urządzenie zostało zamontowane na wylotku ula, a~następnie podłączono je przewodem USB z~laptopem.
Odczekano około godziny, by ruch pszczół uspokoił się po zakłóceniu.
Następnie uruchomiono zapis danych: urządzenie co sekundę przesyłało do komputera aktualne stany wszystkich swoich liczników, które były zapisywane do pliku CSV.
System pracował przez około 1~godzinę 20 minut, po czym pobrano z~niego zapisane dane.
Uzyskane przebiegi czasowe przedstawione zostały na rysunku \ref{fig:real-bee-count}.
\begin{figure}[htb]
    \centering
    \includegraphics{tex/img/real-bee-count.pdf}
    \caption{Przebiegi czasowe liczników pszczół zebrane podczas pracy testowanego urządzenia w~warunkach rzeczywistych.}
    \label{fig:real-bee-count}
\end{figure}

Na wykresach pojedynczych liczników widoczna jest mocna preferencja pszczół przy wyborze tunelu: niektórymi prawie wyłącznie wchodzą do ula, a~innymi głównie wychodzą -- jest to zgodne z~tym, co było widoczne na ręcznie anotowanych danych z~rysunku z~innej pasieki (rysunek \ref{fig:exp1-populacja}).
Przez większość czasu trwania eksperymentu łączna liczba pszczół w~ulu malała.
Wynika to najpewniej z~opuszczania ula przez pszczoły zbieraczki -- odlatują one w~dużych falach, by następnie w~podobnym czasie powrócić do gniazda.
Uzyskane rezultaty przypominają dane, które zebrano wcześniej ręcznie: zarówno pod kątem kształtu przebiegów i~zależności między nimi, jak i~ogólnej intensywności ruchu pszczół.
Można wnioskować, że stworzony system zadziałał prawidłowo poza laboratorium, w~warunkach rzeczywistych.
Uruchomienie urządzenia na dłuższym horyzoncie czasowym może zapewnić pszczelarzom głębszy wgląd w~zdrowie i~stan kolonii, zapewniając dokładną estymację natężenia ruchu pszczół na wylotku ula.
Możliwości oceny chwilowego rozmiaru populacji pozostają jednak ograniczone ze względu na występujący w~systemie błąd zliczania.
% \clearpage
\section{Integracja z systemem telemetrii}


Tutaj o połączeniu bezprzewodowym i integracji z serwerem influxdb i dashboard w grafanie
\clearpage
\section{Podsumowanie}



Tutaj o tym, że sygnał różnicowy z dwóch kondensatorów daje odporność na zmienne warunki atmosferyczne, na niestacjonarność systemu i ogólnie na słabą kalibrację.

Podziękowania

%---------------
% Bibliografia
%---------------
\cleardoublepage % Zaczynamy od nieparzystej strony
\printbibliography
\clearpage

% Wykaz symboli i skrótów.
% Pamiętaj, żeby posortować symbole alfabetycznie
% we własnym zakresie. Makro \acronymlist
% generuje właściwy tytuł sekcji, w zależności od języka.
% Makro \acronym dodaje skrót/symbol do listy,
% zapewniając podstawowe formatowanie.
\acronymlist
\acronym{LED}{Light-Emitting Diode}
\acronym{PCB}{Printed Circuit Board}
\acronym{GPIO}{General-Purpose Input/Output}
\acronym{EXTI}{External Interrupt}
\acronym{USB}{Universal Serial Bus}
\acronym{UART}{Universal Asynchronous Receiver/Transmitter}
\acronym{SPI}{Serial Peripheral Interface}
\acronym{I$ ^2$C}{Inter-Integrated Circuit}
\acronym{SRAM}{Static Random-Access Memory}
\acronym{PC}{Personal Computer}
\acronym{CSV}{Comma-Separated Values}
\acronym{EEPROM}{Electrically Erasable Programmable Read-Only Memory}
\acronym{FSM}{Finite-State Machine}
\acronym{BST}{Binary Search Tree}
\acronym{RAM}{Random-Access Memory}

\vspace{0.8cm}

%--------------------------------------
% Spisy: rysunków, tabel, załączników
%--------------------------------------
\pagestyle{plain}

\listoffigurestoc    % Spis rysunków.
\vspace{1cm}         % vertical space
\listoftablestoc     % Spis tabel.
\vspace{1cm}         % vertical space
% \listofappendicestoc % Spis załączników

%-------------
% Załączniki
%-------------

% Obrazki i tabele w załącznikach nie trafiają do spisów
% \captionsetup[figure]{list=no}
% \captionsetup[table]{list=no}

% Załącznik 1
% \clearpage
% \appendix{Nazwa załącznika 1}
% \lipsum[1-3]
% \begin{figure}[!h]
% 	\centering \includegraphics[width=0.5\linewidth]{logopw2.png}
% 	\caption{Obrazek w załączniku.}
% \end{figure}
% \lipsum[4-7]

% Załącznik 2
% \clearpage
% \appendix{Nazwa załącznika 2}
% \lipsum[1-2]
% \begin{table}[!h] \centering
%     \caption{Tabela w załączniku.}
%     \begin{tabular} {| c | c | r |} \hline
%         Kolumna 1       & Kolumna 2 & Liczba \\ \hline\hline
%         cell1           & cell2     & 60     \\ \hline
%         \multicolumn{2}{|r|}{Suma:} & 123,45 \\ \hline
%     \end{tabular}
% \end{table}
% \lipsum[3-4]

% Używając powyższych spisów jako szablonu,
% możesz dodać również swój własny wykaz,
% np. spis algorytmów.

\end{document} % Dobranoc.
