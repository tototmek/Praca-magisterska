\clearpage
\section{Budowa urządzenia}
\subsection{Założenia projektu}
Aby możliwe było wykorzystanie urządzenia zliczającego pszczoły do diagnostyki ula, musi ono spełniać szereg kryteriów.
Po pierwsze, naturalne funkcjonowanie kolonii nie może zostać w znaczącym stopniu zakłócone. Należy zadbać, by ruch pszczół został zaburzony w minimalnym stopniu, a~wentylacja gniazda nie była zbytnio ograniczona.
Ponadto, istotna jest odporność systemu na warunki atmosferyczne i~aktywność pszczół: m.in. na zmiany nasłonecznienia, wilgotności, zbieranie się pyłku.

Urządzenie ma dostarczać informację o chwilach wejścia/wyjścia, a także kierunku poruszania się pszczoły.
Wyprodukowanie czujnika nie może mimo to wymagać trudno dostępnych lub drogich komponentów, aby możliwe było intensywne iteracyjne prototypowanie i~ewentualne szerokie wykorzystanie stworzonego systemu.

\subsection{Prototypowy czujnik pojemnościowy}
By spełnić wszystkie wymienione w poprzedniej sekcji założenia, stworzono autorski układ wykrywający pszczoły oparty o~czujnik pojemnościowy.
  \subsubsection{Zasada działania}
Pojemność elektryczna kondensatora zależy od przenikalności elektrycznej ośrodka między jego okładkami. W przypadku kondensatora płaskiego, pojemność opisywana jest wzorem:
\[
C = \frac{\varepsilon_r\,\varepsilon_0\,S}{d}
\]
gdzie $C$~--~pojemność kondensatora, $\varepsilon_r$~--~względna przenikalność elektryczna ośrodka, $\varepsilon_0$~--~przenikalność elektryczna próżni, $S$~--~pole powierzchni okładki, $d$~--~odległość między okładkami \cite{efizyka-kondensator}.

Przenikalność elektryczna powietrza wynosi $\varepsilon_r = 1.00$ \cite{Hector1936}. Pszczoła nie jest jednorodnym ośrodkiem i jej parametry elektryczne nie były nigdy mierzone, jednak w znacznej części składa się ona z wody – substancji o wysokim $\varepsilon_r = 80.2$ (przy $20^\circ$C) \cite{Archer1990}.
W literaturze proponowana jest arbitralna wartość przenikalności elektrycznej pszczoły równa $\varepsilon_r = 50$, która znajduje pośrednie potwierdzenie w eksperymentach \cite{Campbell2005}.

Wprowadzenie pszczoły pomiędzy okładki kondensatora spowoduje zatem wzrost średniej przenikalności elektrycznej ośrodka, zwiększając tym samym pojemność układu. Na rysunku \ref{fig:bee-in-capacitor} przedstawiony został układ kondensatora oraz teoretyczny wykres pojemności elektrycznej w zależności od położenia przemieszczającej się pszczoły. Na podstawie kształtu przebiegu pojemności określić można momenty wejścia i opuszczenia kondensatora (odpowiednio $t_{\mathrm{enter}}$ i $t_{\mathrm{leave}}$ na rysunku).

\begin{figure}
    \centering
    \includegraphics{tex//img/bee-in-capacitor.pdf}
    \caption{Schemat układu kondensatora z poruszającą się ruchem jednostajnym pszczołą oraz odpowiadający przebieg pojemności elektrycznej $C_1$.}
    \label{fig:bee-in-capacitor}
\end{figure}

Przedstawiony układ nie umożliwia natomiast określenia kierunku poruszania się pszczoły w czujniku. Aby zapewnić tę funkcję układ rozszerzony jest o drugi kondensator. Schemat takiego układu został przedstawiony na rysunku \ref{fig:bee-in-c2}. Wyjście czujnika stanowi sygnał $C_1 - C_2$. Założono, że kondensatory są identyczne, a więc gdy nie znajduje się w nich pszczoła, to $C_1 - C_2 = 0$. Ruch pszczoły między okładkami generuje na wyjściu bipolarny, środkowosymetryczny impuls, który  daje się zidentyfikować na podstawie kształtu, pozwalając na wykrycie pszczół w czujniku, oraz pozwala określić kierunek ruchu pszczoły na podstawie kolejności występowania piku dodatniego i ujemnego. Na rysunku \ref{fig:bee-in-minus-c2} przedstawiono sytuację, w której pszczoła porusza się w odwrotnym kierunku.

\begin{figure}
    \centering
    \includegraphics{tex//img/bee-in-c2.pdf}
    \caption{Schemat i przebieg $C_1 - C_2$ dla układu dwóch kondensatorów.}
    \label{fig:bee-in-c2}
\end{figure}

\begin{figure}
    \centering
    \includegraphics{tex//img/bee-in-minus-c2.pdf}
    \caption{Schemat i przebieg $C_1 - C_2$ dla układu dwóch kondensatorów -- pszczoła porusza się w odwrotnym kierunku.}
    \label{fig:bee-in-minus-c2}
\end{figure}

Przedstawione ilustracje mają na celu wyłącznie prezentację podstawowej zasady działania czujnika pojemnościowego, a podczas ich tworzenia pominięte zostało wiele zjawisk fizycznych.
W praktyce, kondensatory o płaskich okładkach nie sprawdzają się takim zastosowaniu -- są podatne na zewnętrzne zakłócenia, a także muszą być znacznie oddalone od siebie by uniknąć wzajemnego zakłócania \cite{Campbell2005}.
Mimo to, rozwiązania znalazły wykorzystanie w pracach naukowych, na przykład urządzenie \textit{BeeCheck} \cite{Bermig2020}, kompensujące wady płaskich okładek wykorzystaniem 7 kondensatorów w czujniku.

Znacznie skuteczniejszym podejściem jest wykorzystanie okładek pierścieniowych. Na rysunku \ref{fig:tunnel-concept} przedstawiona została ogólna idea tego rozwiązania. Na czujnik składa się rurka z izolatora, przez którą przechodzi owad, oraz trzy pierścienie z przewodnika.
Konstrukcja ta realizuje taki sam układ elektryczny jak analizowano poprzednio, przy czym kondensatory $C_1$ i $C_2$ mają wspólną okładkę (środkowy pierścień – wyprowadzenie $C$). Kondensator $C_1$ znajduje się między wyprowadzeniami $L$ i $C$, natomiast kondensator $C_2$ między $R$ i $C$.
Ułożenie to, pozwalając na równoczesne ładowanie obu kondensatorów, umożliwi pomiar różnicy pojemności $C_1 - C_2$.

\begin{figure}
    \centering
    \includegraphics{tex//img/tunnel-concept.pdf}
    \caption{Konstrukcja czujnika opartego na elektrodach pierścieniowych.}
    \label{fig:tunnel-concept}
\end{figure}

Działanie czujnika tego typu zostało zasymulowane numerycznie przez Campbell wraz z zespołem \cite{Campbell2005}. Przyjęto: średnicę wewnętrzną tunelu 15 mm, okładki kondensatora wykonane z drutu o średnicy 1.7 mm, odległość między okładkami 7 mm. Pszczołę zamodelowano jako cylinder o długości 12 mm i średnicy 6mm. Wyniki tej symulacji przedstawione zostały na rysunkach \ref{fig:campbell-symulacja-model} i \ref{fig:campbell-symulacja-wynik}.
Model czujnika zakładał dodatkowe obudowanie całości stalowym ekranem i zastosowanie polistyrenowych zaślepek. Na rysunku \ref{fig:campbell-symulacja-model} widać, że pszczoła (symulowana jako cylinder o~$\varepsilon_r=50$) wchodząca do tunelu powoduje niesymetryczne rozłożenie pól elektrycznych w kondensatorach.
Przejście pszczoły skutkuje pojawieniem się na wyjściu czujnika asymetrycznych, bipolarnych impulsów (rysunek \ref{fig:campbell-symulacja-wynik}), podobnych do tych na przebiegach z rozważań teoretycznych (rysunki \ref{fig:bee-in-c2}, \ref{fig:bee-in-minus-c2}). W symulacji przetestowano dodatkowo odpowiedź układu na pszczoły różnych rozmiarów (impulsy $a$, $b$ na rysunku \ref{fig:campbell-symulacja-wynik}). Zaobserwować można pewną różnicę w wyglądzie wygenerowanych przebiegów, jednak ogólny kształt nadal pozwala na rozpoznanie momentu oraz kierunku przejścia pszczoły.

\begin{figure}
    \centering
    \includegraphics{tex//img/campbell-symulacja-model.pdf}
    \caption{Przekrój poprzeczny czujnika oraz linie ekwipotencjalne -- symulacja przeprowadzona przez Campbell wraz z zespołem. \cite{Campbell2005}}
    \label{fig:campbell-symulacja-model}
\end{figure}

\begin{figure}
    \centering
    \includegraphics{tex//img/campbell-symulacja-wynik.pdf}
    \caption{Impulsy wygenerowane przez pszczołę przechodzącą przez czujnik. Wyniki symulacji. Różnica pojemności kondensatorów była przetwarzana na napięcie przez liniowy przetwornik. (a)~--~duża pszczoła opuszczająca ul; (b)~--~mała pszczoła wchodząca do ula. \cite{Campbell2005}}
    \label{fig:campbell-symulacja-wynik}
\end{figure}

  \subsubsection{Układ pomiarowy}
Sygnał wyjściowy omawianego czujnika pojemnościowego jest trudny do zmierzenia. Pojemności kondensatorów $C_1$, $C_2$ wynoszą zaledwie ok. 0,5 pF \cite{Campbell2005}, a przy pojawieniu się pszczoły zwiększają się tylko o kilka--kilkanaście procent.
Pomiar tak małych pojemności wymaga specjalistycznego sprzętu oraz niezwykłej dbałości podczas projektowania układu.
Pojemności pasożytnicze ścieżek na płytce drukowanej mogą być przy słabym projekcie nawet o rząd wielkości wyższe niż pojemność badanego kondensatora.
Szczegółowe założenia, którymi kierowano się przy projekcie PCB by zminimalizować ich wpływ, zostały szczegółowo opisane w sekcji \ref{pcb}.

W literaturze znaleźć można przykładowe układy pomiarowe, które działają na zasadzie różnicy przesunięcia fazowego. Wspólna okładka kondensatorów ekscytowana jest sygnałem okresowym (generowanym z pomocą układu Intersil ICL8038 \cite{Campbell2005} lub NE555P \cite{mathematastic}). Kondensatory czujnika połączone są z masą układu przez rezystory, tworząc razem filtr górnoprzepustowy przesuwający fazę sygnału w zależności od pojemności. Przesunięte sygnały są następnie odejmowane z pomocą precyzyjnego wzmacniacza różnicowego.
Następnie następuje demodulacja i wzmocnienie sygnału. W efekcie, wyjście układu stanowi sygnał o napięciu proporcjonalnym do różnicy pojemności obu kondensatorów. 

Taki sposób pomiaru został z powodzeniem zastosowany w urządzeniach zliczających pszczoły, jednak jest bardzo kosztowny. Konieczne jest użycie układów scalonych o wysokiej precyzji – szacunkowy koszt układów scalonych potrzebnych do obsłużenia jednej pary kondensatorów to 231 zł\footnote{Ceny na dzień 09.10.2025 za: ICL8038, AD620, AD630, LF411.}. Całe urządzenie ma się składać z ośmiu równoległych tuneli, łączna cena układów scalonych wyniosłaby 1848 zł znacznie przekraczając limit kosztu całego urządzenia.

Alternatywnym podejściem jest zastosowanie dedykowanego scalonego układu przetwornika pojemnościowo-napięciowego.
Przykładem takiego układu odpowiedniego do małych pojemności jest AD7746, zastosowany w pracy Paula Perraulta \cite{Perrault2016}.
Jest on przetwornikiem dwukanałowym, wystarczy zatem jeden na oba kondensatory w pojedynczym czujniku \cite{ad7746}. Koszt zakupu jednego układu scalonego to 67 zł (sumarycznie 536 zł na 8~tuneli), co jednak wciąż przekracza zakładany budżet.\newline

W ramach niniejszej pracy opracowany został prosty i niezwykle tani sposób akwizycji sygnału z czujnika pojemnościowego. Jakość pomiaru jest znacznie niższa niż w opisanych wcześniej rozwiązaniach, lecz uzyskane przebiegi czasowe pozwalają skutecznie wykrywać ruch pszczół i określać jego kierunek.
Schemat proponowanego układu przedstawiony jest na rysunku \ref{fig:uklad-pomiarowy}.
Jego działanie opiera się na różnicy czasu ładowania kondensatorów o różnych pojemnościach.

\begin{figure}
    \centering
    \includegraphics{tex//img/ladowanie-kondensatora.pdf}
    \caption{Ładowanie kondensatora: schemat układu i przebiegi napięć.}
    \label{fig:ladowanie kondensatora}
\end{figure}


Rozważmy układ pojedynczego kondensatora (rysunek \ref{fig:ladowanie kondensatora}). Odpowiedź $u_C(t)$ na skok napięcia $u$ od 0 do $U$ dana jest jako:
\[
u_C(t) = U\,\left(1-e^{-\frac{t}{\tau}}\right)
\]

gdzie $\tau = R\,C$ --- stała czasowa układu \cite{ladowanie-kondensatora}. Przykładowe przebiegi odpowiedzi skokowej przedstawione zostały na rysunku \ref{fig:ladowanie kondensatora}. Większa stała czasowa wiąże się z wolniejszym narastaniem napięcia na kondensatorze. 
Przyjmując pewne napięcie progowe $U_{\mathrm{prog}}$, możemy określić czas ładowania kondensatora jako czas od skoku $u$ do przekroczenia $U_{\mathrm{prog}}$ przez $u_C$.
Ponieważ $u_C(t)$ jest funkcją rosnącą, to dla stałych czasowych $\tau_1<\tau_2$ odpowiadające im czasy ładowania spełniają zależność $t_1 < t_2$.
Odpowiedni dobór $U_\mathrm{prog}$ ma wpływ na jakość działania systemu: przy niskim progu $t_1$ oraz $t_2$ będą niewielkie i bardzo zbliżone do siebie. Wysoki próg zwiększa podatność na szumy w pomiarze $u_C$. Wybór konkretnej wartości $U_\mathrm{prog}$ został szerzej opisany dalej.

W układzie czujnika każdy z kondensatorów ładowany jest przez taką samą rezystancję –-stałe czasowe podukładów zależą wyłącznie od pojemności elektrycznej samych kondensatorów. Metoda wykrywania pszczół stworzona w ramach niniejszej pracy nie została zatem oparta bezpośrednio na różnicy pojemności kondensatorów, tylko na różnicy w czasie ich ładowania.

Wyzwaniem technicznym jest sam pomiar czasu ładowania kondensatora o tak niewielkiej pojemności. Przyjmując, jak wcześniej, orientacyjną pojemność kondensatora równą 0,5 pF, wyznaczyć można szacowaną stałą czasową układu (z rezystorem 1M$\Omega$):
\[
\tau = R\cdot C = 1\mathrm{M}\Omega\cdot0.5\mathrm{pF}
 =  
10^6\Omega\cdot0.5\cdot10^{-12}\mathrm{F}
 = 
2\cdot10^{-6}\mathrm{s}
 = 
2 \mu\mathrm{s}
\]
\begin{figure}
    \centering
    \includegraphics{tex//img/uklad-pomiarowy.pdf}
    \caption{Schemat układu pomiarowego.}
    \label{fig:uklad-pomiarowy}
\end{figure}
Jak widać, stała czasowa układu przyjmuje wartości rzędu pojedynczych mikrosekund. Precyzyjny pomiar takich czasów jest zasadniczo niemożliwy z wykorzystaniem sprzętu o niskim koszcie.
Z tego powodu, w niniejszej pracy zastosowana została pewna sztuczka: ładowanie kondensatora mierzone jest nie w jednostkach czasu, ale w liczbie wykonań pojedynczych instrukcji mikrokontrolera monitorującego stan wyjść czujnika. Działanie to formalnie nie daje faktycznego pomiaru czasu, jednak wynik pozwala się jako taki pomiar zastosować – co zostanie udowodnione w dalszych częściach pracy.
Na rysunku \ref{fig:uklad-pomiarowy} przedstawione zostało połączenie czujnika z mikrokontrolerem, natomiast pseudokod \ref{alg:akwizycja} pokazuje działanie algorytmu akwizycji danych.

\begin{algorithm}
\caption{Algorytm akwizycji pomiaru}
\label{alg:akwizycja}
\begin{algorithmic}
\State $n_L \gets 0$\
\State $n_R \gets 0$\
\State $s_L \gets 0$\
\State $s_R \gets 0$\
\State \texttt{write\_gpio(}Charge\texttt{, 1)}\
\While{$s_L = 0$ \textbf{or} $s_R = 0$ }
    \State $s_L \gets $\texttt{read\_gpio(}$L$\texttt{)}\
    \State $s_R \gets $\texttt{read\_gpio(}$R$\texttt{)}\
    \If{$s_L$ = 0}
        \State $n_L \gets n_L + 1$
    \EndIf
    \If{$s_R$ = 0}
        \State $n_R \gets n_R + 1$
    \EndIf
\EndWhile
\State \texttt{write\_gpio(}Charge\texttt{, 0)}\
\State \texttt{return} $n_L - n_R$
\end{algorithmic}
\end{algorithm}

Algorytm inicjalizuje dwa liczniki iteracji odpowiadające dwóm kondensatorom: $n_l$ oraz $n_r$. Następnie na pin \textit{Charge} wystawiany jest stan wysoki, co rozpoczyna ładowanie kondensatorów.
Program iteracyjnie inkrementuje liczniki dopóki piny GPIO $L$ oraz $R$ nie zostaną podniesione do stanu wysokiego.
Stan wysoki występuje po przekroczeniu przez napięcie kondensatora progu jedynki logicznej mikrokontrolera – stanowi on w tym przypadku $U_\mathrm{prog}$ układu.
Wartości progowe są dobierane tak, by minimalizować czas odpowiedzi pinu, ale również podatność na szumy. Optymalizowane przez projektantów mikrokontrolerów kryteria są podobne do tych, które musi spełniać odpowiednio dobrany próg $U_\mathrm{prog}$, można się zatem spodziewać, że system wykorzystujący cyfrowe piny GPIO będzie działał prawidłowo.
Po naładowaniu kondensatorów i ustaleniu $n_L$ oraz $n_R$ pin \textit{Charge} ponownie ustawiany jest do stanu niskiego. Przed wykonaniem kolejnego pomiaru kondensatory muszą zostać rozładowane, nie wymaga to jednak rozbudowywania układu, ponieważ zastosowany rodzaj kondensatora szybko rozładowuje się samoistnie.
Należy jednak pamiętać, aby na kolejnych etapach projektu wprowadzić ograniczenie częstotliwości próbkowania, aby dać ładunkom elektrycznym czas na rozproszenie.

Zaprezentowany algorytm akwizycji oferuje rozdzielczość pomiaru zależną od czasu wykonania instrukcji znajdujących się wewnątrz pętli \textit{while}.
Najdłuższą operacją jest odczyt stanu pinu (\texttt{read\_gpio}) – dla zastosowanego mikrokontrolera ESP32-C3 może on trwać $0.05 - 0.17\,\mu\mathrm{s}$ (w zależności od implementacji) \cite{ESP32C3Datasheet}\cite{microcontroller-benchmarks}. Nawet przy konieczności dwukrotnego wykonania instrukcji, rozdzielczość taka jest wystarczająca na potrzeby tworzonego urządzenia.

Alternatywnym podejściem, niewykorzystanym w niniejszej pracy z braku wyraźnej potrzeby, jednak zapewniającym znacznie wyższą rozdzielczość byłoby wykorzystanie zewnętrznych przerwań mikrokontrolera. Modyfikacja przedstawionego algorytmu polega na usunięciu odczytu GPIO z pętli \textit{while}, zamiast czego $s_L$ oraz $s_R$ ustawiane są na wartość 1 w ramach obsługi przerwań EXTI wynikających z wykrycia zboczy narastających na pinach $L$ i $R$.

  
  \subsubsection{Realizacja sprzętowa}


\begin{figure}
    \centering
    \includegraphics[width=0.8\textwidth]{tex/img/prototyp.png}
    \caption{Prototypowy czujnik pojemnościowy.}
    \label{fig:prototyp}
\end{figure}

W celu przetestowania działania opisanego systemu zbudowana została prototypowa konstrukcja czujnika (rysunek \ref{fig:prototyp}).
Tunel czujnika wykonany został z rurki z przezroczystego tworzywa sztucznego, natomiast okładki wykonano z obręczy wygiętych z drutu miedzianego i zlutowanych w miejscu łączenia.
Wyprowadzenia okładek kondensatora zostały wykonane z izolowanych przewodów o długości ok. 15 cm.
W tabeli \ref{tab:prototyp} zawarte zostały parametry opracowanej konstrukcji.
Obrane wymiary oparte zostały na wynikach optymalizacji podobnego czujnika w pracy Campbell wraz z zespołem, w której przeprowadzono symulację elektrostatyczną, by następnie dobrać średnice $s$ i rozstaw okładek $d$, tak by maksymalizować amplitudę sygnału generowanego przez pszczołę w czujniku \cite{Campbell2005}.
Wypracowane w tej pracy kryteria to:
\begin{enumerate}
    \item \(\frac{1}{2}\,s+t < d <s + 2\,t\) -- rozstaw okładek powinien być większy od promienia okładki, ale nie większy od średnicy;
    \item $g$ -- średnica drutu, z którego wykonane są okładki powinna być jak największa -- na ile pozwolą inne ograniczenia konstrukcyjne.
\end{enumerate}
\begin{figure}
    \centering
    \includegraphics{tex/img/prototyp-tech.pdf}
    \vspace{2em}
    \centering
    \begin{tblr}{cl|r}
        \textbf{Oznaczenie} & \textbf{Opis} & \textbf{Wartość} [mm] \\
        \hline
       $\textbf{\textit{l}}$ & Długość tunelu & 120  \\
       $\textbf{\textit{D}}$ & Odległość do 1. okładki & 8 \\
       $\textbf{\textit{d}}$ & Rozstaw okładek & 9 \\
       $\textbf{\textit{g}}$ & Średnica drutu okładki & 1.5 \\
       $\textbf{\textit{t}}$ & Grubość ścianki tunelu & 0.85 \\
       $\textbf{\textit{s}}$ & Średnica wewn. tunelu & 14.3 \\
    \end{tblr}
    \captionof{table}{Wymiary prototypowego tunelu}
    \label{tab:prototyp}
\end{figure}

W ramach niniejszej pracy założono, że wewnętrzna średnica tunelu musi wynosić około 14 mm, aby zapewnić swobodny ruch pszczół, a przy tym uniemożliwić równoległe poruszanie się kilku osobników obok siebie.
Średnica zewnętrzna zastosowanej rurki z tworzywa sztucznego to 16 mm, co daje promień okładki wynoszący 8 mm. Dla spełnienia wyżej wymienionego kryterium obrano rozstaw okładek $d=9$ mm.
Do wykonania okładek wybrano drut miedziany o średnicy 1.5 mm.

\begin{figure}
    \centering
    \includegraphics{tex/img/uklad-prototyp.pdf}
    \caption{Schemat układu elektronicznego prototypowego tunelu.}
    \label{fig:uklad-prototyp}
\end{figure}


Na uniwersalnej płytce prototypowej stworzono układ elektroniczny zgodny tym przedstawionym na rysunku \ref{fig:uklad-pomiarowy}. Na potrzeby przetestowania prototypu pominięto jeden z kondensatorów w tunelu i do mikrokontrolera podłączono jedynie okładkę lewego kondensatora. Zastosowany został wspominany wcześniej mikrokontroler ESP32-C3 na płytce deweloperskiej ESP32-C3 Super Mini \cite{ESP32C3Datasheet}. Moduł ten wyposażony jest w konwerter USB-UART umożliwiający komunikację urządzenia z komputerem z wykorzystaniem złącza USB. Na mikrokontroler zaimmplementowana została uproszczona (mierząca tylko jedną z bramek) wersja algorytmu \ref{alg:akwizycja}:


\inputminted[label={prototype/main.cpp}, linenos, frame=single, framesep=2em]{cpp}{code/prototype-main.cpp}



  \subsubsection{Testy}
  
\subsection{Układ elektroniczny}
  \subsubsection{Dobór komponentów}
  \subsubsection{Realizacja płytki PCB} \label{pcb}
\subsection{Oprogramowanie mikrokontrolerów}
  \subsubsection{Algorytm akwizycji}
  \subsubsection{Komunikacja między mikrokontrolerami}
  \subsubsection{Komunikacja ze światem zewnętrznym}
\subsection{Testy}
\subsection{Modyfikacja układu}
\subsection{Realizacja obudowy}