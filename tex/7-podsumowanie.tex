\clearpage
\section{Podsumowanie}

Cel niniejszej pracy stanowiło stworzenie systemu czujników umożliwiającego zliczanie pszczół w~ulu.
Na podstawie przeglądu powiązanej literatury zdecydowano się wykorzystać czujniki pojemnościowe, których stosowanie zyskuje w~ostatnich latach popularność.
Przyjęto, że konstruowane urządzenie ma przyjąć formę szeregu tuneli z~bramkami pojemnościowymi, montowanego na wejściu do ula.
Realizacja zadania wymagała rozwiązania dwóch głównych problemów:
\begin{enumerate}
    \item zaprojektowania czujników pojemnościowych i~metody akwizycji ich sygnałów,
    \item opracowania algorytmu umożliwiającego wykrywanie pszczół na podstawie wyjść czujników pojemnościowych.
\end{enumerate}

W pierwszym etapie pracy zbudowano prototyp czujnika, bazując na konstrukcjach dostępnych w~literaturze.
Opracowano dla niego uproszczoną metodę akwizycji sygnału, nie wymagającą kosztownych komponentów ani kalibracji.
W ramach eksperymentu z~modelem pszczoły potwierdzono prawidłowe działanie prototypowej konstrukcji i~metody akwizycji.
Kolejnym krokiem w~pracy była realizacja sprzętowa urządzenia wykorzystujące czujniki stworzone na bazie prototypu, dostosowane do gatunku pszczół \textit{Apis mellifera}.
Dobrano mikrokontroler, a~następnie zaprojektowano płytkę PCB z~układem ośmiu czujników.
Po iteracji projektu, układ działał w~pełni poprawnie, co potwierdzono kolejnym badaniem laboratoryjnym na modelu pszczoły.
Uzyskiwane przebiegi wyjść czujników były bardzo podobne do literaturowych, uzyskanych przez symulację elektrostatyczną.

Przed przejściem do etapu tworzenia algorytmu detekcji pszczół, konieczne było zebranie dużej ilości danych przykładowych, które miały umożliwić ocenę jakości działania kolejnych rozwiązań.
W tym celu, przeprowadzone zostały pierwsze testy pracy urządzenia w~warunkach rzeczywistych: odwiedzono dwie różne pasieki, w~których zebrano dane generowane przez urządzenie działające z~żywymi pszczołami, wraz z~referencyjnym nagraniem wideo.
Kilka godzin uzyskanych danych zostało poddane ręcznej anotacji, w~ramach której oznaczono chwile przechodzenia pszczół przez tunele czujników.

Z obszernym zbiorem oetykietowanych danych, można było przejść do projektu algorytmu detekcji.
W ramach niniejszej pracy zaproponowane zostały dwa rozwiązania, które wykazywały duży potencjał, zachowując przy tym niską złożoność obliczeniową: wykrywanie sekwencji progowanego sygnału automatem stanowym, oraz korelacja krzyżowa z~wzorcem wykrywanego impulsu.
Liczne konfiguracje obu algorytmów zostały uruchomione dla zbioru danych przykładowych, a~ich wyjścia porównano z~ręcznie oznaczonymi etykietami.
Najwyższą zgodność osiągniął algorytm oparty na automacie stanowym.
Wybraną metodę detekcji zaimplementowano dla platformy sprzętowej urządzenia, po uprzedniej koniecznej optymalizacji najważniejszych fragmentów.
Skuteczność opracowanego oprogramowania dla nowych danych potwierdzono w~ostatnim eksperymencie laboratoryjnym.
Stworzony system ponownie uruchomiono na prawdziwym ulu -- tym razem z~jego pełną funkcjonalnością automatycznego zliczania pszczół.
Zebrane w~ten sposób dane zostały zawarte i~omówione w~niniejszej pracy.

\newpage

Przeprowadzone badania pozwoliły na wyciągnięcie następujących wniosków:
\begin{enumerate}
    \item Czujniki pojemnościowe prezentowanego typu działają prawidłowo, a~w połączeniu z~opracowaną metodą akwizycji i~przetwarzaniem generują dobrej jakości sygnał wyjściowy.
    Impulsy powstające przy ruchu pszczół w~czujniki są wyraźnie widoczne i~bez problemów mogą zostać wykryte automatycznie.
    Zastosowanie w~roli sygnału wyjściowego różnicy między pojemnościami dwóch kondensatorów zapewnia rozwiązaniu odporność na zmienne warunki środowiska, niestacjonarność systemu, a~także niweluje konieczność kalibracji.
    Niezwykła prostota proponowanego układu akwizycji znacząco obniża koszt wykonania urządzenia;
    \item Montaż urządzenia na wylotku ula wydaje się mieć znacznie mniejszy negatywny wpływ na zachowanie pszczół, niż się spodziewano.
    Ruch owadów przez tunele rozpoczynał się praktycznie od razu; pszczoły nie próbowały także omijać urządzenia i~przedostawać się do ula bokiem.
    Ograniczenie ruchu powietrza też nie wydało się być problemem -- nawet przy dłuższej pracy systemu nie pojawiało się zbyt wiele osobników wentylujących wejście ula;
    \item Opracowany algorytm detekcji działa poprawnie, kiedy pszczoły w~tunelu poruszają się typowo.
    Zdarzają się błędy wynikające z~niestandardowych zachowań pszczół, takich jak: grupowanie w~tunelu, zatrzymywanie w~czujniku.
    Zachowania te okazują się występować częściej niż zakładano, przez co dokładność systemu jest niższa, niż oczekiwano.
    W~aktualnej wersji, urządzenie nie pozwala na skuteczne szacowanie aktualnego stanu populacji ula.
    Zapewnia natomiast możliwość oceny chwilowego natężenia ruchu pszczół na jego wejściu, co również stanowi ważną metrykę świadczącą o~zdrowiu i~sile kolonii.
\end{enumerate}

Udowodniono, że opracowana technika automatycznego zliczania pszczół działa poprawnie.
Znacząco obniżono koszt budowy urządzenia względem innych prac, co może pozwolić na stosowanie systemu na większą skalę również w~mniejszych pasiekach.
Kontynuacja pracy nad stworzonym rozwiązaniem da szanse ograniczyć ilość występujących błędów.
Możliwe kierunki rozwoju to:
\begin{enumerate}
    \item Zastosowanie większej liczby bramek pojemnościowych w~każdym tunelu urządzenia -- dodatkowe dane mogą pozwolić lepiej rozróżniać atypowe zachowania pszczół;
    \item Wykorzystanie bardziej zaawansowanych metod detekcji pszczelich impulsów, które byłyby zdolne do interpretacji przebiegów z~anomaliami;
    \item Opracowanie metody szacowania całkowitej populacji ula opartej na fuzji danych z~modelem -- rozszerzenie analizy sygnałów o~model probabilistyczny uzależniający natężenie ruchu pszczół od ich liczby w~gnieździe.
    Połączenie danych z~dwóch źródeł: prostego zliczania wejść i~wyjść, wraz z~częstotliwością ruchu na wlotku mogłoby zapewnić estymację stanu populacji znacznie mniej podatną na akumulację występujących w~systemie błędnych zliczeń.
\end{enumerate}


\section*{Podziękowania}
Dziękuję dr. Jakubowi Gąbce za podzielenie się ekspercką wiedzą dotyczącą pszczół, oraz p. Andrzejowi Bielackiemu i p. Wojciechowi Sokołowi za udostepnienie swoich uli do eksperymentów.
