\clearpage
\section{Algorytm detekcji} \label{chapter:algorytm}

Opisany w poprzedniej części pracy system czujników generuje sygnały wyjściowe, na przebiegach czasowych których widoczne są impulsy wywołane ruchem pszczół.
W celu umożliwienia wykorzystania stworzonego urządzenia do automatycznego zliczania pszczół w ulu, konieczne jest zaprojektowanie algorytmu wykrywającego te impulsy.
Przyjęte zostały następujące założenia:
\begin{enumerate}
    \item Algorytm ma być uruchomiony na mikrokontrolerze wchodzącym w skład urządzenia, co pozwoli uniknąć konieczności rozszerzenia systemu o dodatkowy węzeł, którego zadaniem byłoby jedynie wykonywanie obliczeń.
    Wymagać to będzie niskiej złożoności pamięciowej algorytmu oraz niewielkiego narzutu obliczeniowego.
    \item Algorytm ma być wykonywany \textit{on line} -- w każdej iteracji działania urządzenia, lub na buforze z maks. kilku sekund. W ten sposób zagwarantowane będą bieżące dane, oraz brak przerw w akwizycji sygnału z czujników pojemnościowych.
\end{enumerate}


\subsection{Zbiór danych przykładowych}

Pierwszym krokiem, zanim rozpoczęto pracę nad projektem algorytmu, było zebranie dużej ilości przykładowych danych z pracy urządzenia w warunkach rzeczywistych.
Są one niezwykle ważne, pozwolą bowiem do oceny proponowanych w kolejnych sekcjach algorytmów i ich konfiguracji -- posłużą do przeprowadzenia testów \textit{off line}.

  \subsubsection{Testy urządzenia w pasiece}

W celu uzyskania reprezentatywnych danych z pracy urządzenia w środowisku rzeczywistym przeprowadzone zostały eksperymenty w dwóch niezależnych pasiekach, w różnych tygodniach, przy innych warunkach atmosferycznych i aktywności pszczół. Zbierano sygnały $x_i(t)$, dla $i=0\dots7$, czyli wyjściowe przebiegi czasowe wszystkich bramek urządzenia.
Testy prowadzono według następującej metody:
\begin{enumerate}
    \item Wybierano ul, który zostanie wykorzystany do eksperymentu. W obu przypadkach kierowano się największą aktywnością pszczół w całej pasiece, by w zebranych przebiegach wystąpiło jak najwięcej przejść owadów przez czujnik;
    \item Na wylotku wybranego ula umieszczano urządzenie. Zastosowano dwie alternatywne jego pozycje: wewnątrz ula -- zamocowane do dennicy, a także na zewnątrz ula na półce przed wylotkiem.
    Pszczelarze w prowadzonych rozmowach sugerowali, że przy drugiej z wymienionych metod pszczoły mogą być niechętne do wchodzenia do gniazda, jednak w eksperymencie nie zaobserwowano tego zjawiska.
    Wszelkie szczeliny pozostające między urządzeniem a ścianami ula zatkano, aby uniemożliwić pszczołom ominięcie tuneli czujnika.
    \item W obu eksperymentach odczekano około 30 minut, aby pszczoły poruszone zamieszaniem przy montażu urządzenia uspokoiły się, a następnie aby przyzwyczaiły się do nowego obiektu na wylotku.
    \item Na statywie przed ulem umieszczono kamerę wideo, nagrywającą obraz urządzenia. Kadr dobrany został tak, by wyraźnie widoczne były wloty wszystkich tuneli urządzenia.
    Nagranie to będzie później wykorzystane do anotacji danych -- ręczne opisu chwil, w których pszczoły wchodziły i wychodziły.
    \item Urządzenie łączono z portem USB laptopa, aby zapewnić przesył kolejnych próbek sygnałów wyjściowych każdego z czujników pojemnościowych.
    Uruchamiano stworzony uprzednio skrypt Python (plik \texttt{scripts/collect\_data.py} w repozytorium projektu), zapisujący otrzymywane dane do pliku CSV.
    \item W chwili uruchomienia w komputerze akwizycji sygnału, wykonywano przed kamerą klaśnięcie, co pozwoli na późniejszą synchronizację nagrania z zebranymi danymi.
\end{enumerate}
Na rysunku \ref{fig:exp2} przedstawiona jest fotografia prezentująca jeden z przeprowadzonych eksperymentów, natomiast na rysunku \ref{fig:exp1} przedstawiono alternatywne miejsce montażu urządzenia (od środka ula).
W tabeli \ref{tab:test1} podsumowano informacje o przeprowadzonych eksperymentach.
\begin{table}[H]
\centering
\caption{Zbiór danych przykładowych -- podsumowanie eksperymentów}
\begin{tabular}{l|l|l|l}
                       & \multicolumn{1}{c|}{\textbf{Eksperyment 1}} & \multicolumn{1}{c|}{\textbf{Eksperyment 2}} &                                                \\ 
\cline{1-3}
\textbf{Data}          & 20 lipca 2025                               & 26 lipca 2025                               &                                                \\
\textbf{Lokalizacja}   & gmina Miastkowo                             & gmina Choroszcz                             & \multicolumn{1}{c}{\textit{\textbf{Łącznie}}}  \\ 
\cline{4-4}
\textbf{Czas trwania}  & ok. 45 min                                  & ok. 30 min                                  & ok. 1 h 15 min                                 \\
\textbf{Liczba próbek} & 274214                                      & 182435                                      & 456649                                        
\end{tabular}
\label{tab:test1}
\end{table}

Przeprowadzone eksperymenty stanowiły również pierwszą weryfikację działania urządzenia w warunkach rzeczywistych -- przy współpracy z żywymi pszczołami.
Fragment zebranych danych zwizualizowano na wykresie (rysunek \ref{fig:exp1-data}), w celu sprawdzenia czy zawiera prawidłowo wyglądające impulsy wywoływane ruchem pszczół.
Stwierdzono, że system działa prawidłowo: pomimo występującego szumu widoczne są wzory bardzo podobne do wzorcowych przebiegów; warto porównać zaprezentowane przebiegi z tymi na rysunku \ref{fig:campbell-symulacja-wynik}.
\begin{figure}[H]
    \centering
    \includegraphics{tex/img/wizualizacja-testu-z-pasieki.pdf}
    \caption{Przebieg sygnału $x_0(t)$ (wyjścia pierwszego czujnika pojemnościowego) w pierwszych kilkudziesięciu sekundach trwania eksperymentu. Przebieg filtrowany został wygenerowany poprzez nałożenie na dane oryginalne filtru średniej kroczącej o długości okna równej 50.}
    \label{fig:exp1-data}
\end{figure}
\begin{figure}[p]
    \centering
    \includegraphics[width=0.9\linewidth]{tex/img/8-eksperyment-2-downscaled.jpg}
    \caption{Zdjęcie jednego z eksperymentów. Widoczne jest urządzenie zamontowane na wylotku ula, oraz kamera nagrywające jego wylot. Znajdujący się na obrazie fioletowy przewód łączy czujnik z komputerem zapisującym dane.}
    \label{fig:exp2}
\end{figure}
\begin{figure}[p]
    \centering
    \includegraphics[width=0.9\linewidth]{tex/img/7-eksperyment-1.JPG}
    \caption{Zdjęcie prezentujące montaż urządzenia podczas drugiego eksperymentu. Jest ono przybite do dennicy -- pozostałą część ula nastawia się na widoczny na obrazie element.}
    \label{fig:exp1}
\end{figure}

\FloatBarrier
  
  \subsubsection{Ręczna anotacja danych}
Umożliwienie oceny algorytmów z wykorzystaniem zbudowanego zbioru danych wymaga przeprowadzenia jego anotacji.
Polega ona na niezależnym, ręcznym oznaczeniu chwil przechodzenia pszczół przez tunele czujnika.
Podczas testowania algorytmów weryfikowana ma być zgodność generowanych przez nie wyników z tymi właśnie oznaczeniami.
Do anotacji wykorzystane zostały wspomniane wcześniej nagrania wideo z eksperymentów.
W celu przyspieszenia procesu ręcznego przeglądu materiału, a także zredukowania prawdopodobieństwa popełniania błędów, stworzono dedykowany program: wyświetlał on użytkownikowi nagranie, dając możliwość kontroli tempa odtwarzania, a także zautomatyzowanego tworzenia anotacji danych poprzez pojedyncze uderzenia klawiszy (plik \texttt{scripts/manual\_annotation.py}).
Wciśnięcie wybranego przycisku zapisywało dla danej chwili nagrania wystąpienie wejścia/wyjścia pszczoły przez odpowiedni tunel.
Na rysunku \ref{fig:annotation} przedstawiony został zrzut ekranu z programu -- prezentuje on także przykładowy kadr z nagrania eksperymentu.
\begin{figure}[htb]
    \centering
    \includegraphics[width=\linewidth]{tex/img/manual-annotation.png}
    \caption{Zrzut ekranu z programu do ręcznej anotacji danych z eksperymentu.}
    \label{fig:annotation}
\end{figure}
Ręczne etykietowanie całego materiału nagraniowego okazało się wymagać większego nakładu pracy niż początkowo oczekiwano -- ze względu na ograniczenia percepcji operatora programu konieczne było przetwarzanie całej długości nagrania z osobna dla każdego z tuneli, co ośmiokrotnie wydłużyło oczekiwany czas pracy.

Wynikiem niniejszego etapu pracy było powstanie dla każdego z tuneli urządzenia dwóch zbiorów anotacji: $L_i^{\mathrm{we}}$ -- zawierającego chwile $t$, w których pszczoły wchodziły do ula przez tunel $i$; oraz $L_i^{\mathrm{wy}}$ -- zawierającego chwile $t$, w których pszczoły wychodziły z ula przez tunel $i$.
Etykiety w takiej formie porównano z oryginalnymi sygnałami wyjściowymi $x_i(t)$, aby empirycznie potwierdzić, że impulsy generowane przez pszczoły pokrywają się z anotacjami. Na rysunku \ref{fig:anotacja-dane} przedstawiony jest fragment danych, na którym wyraźnie widać, że anotacje są prawidłowe, oraz że nie wystąpiły problemy z synchronizacją danych z czujnika i nagrania. Każdemu z wyraźnie widocznych impulsów odpowiada element zbioru etykiet; ponadto, rodzaj anotacji (wejście/wyjście) jest zgodny z polaryzacją impulsów.
\begin{figure}[htb]
    \centering
    \includegraphics{tex/img/anotacja-dane.pdf}
    \caption{Fragment przebiegu sygnału $x_0$ zestawiony z ręcznie dodaną anotacją.}
    \label{fig:anotacja-dane}
\end{figure}

Na rysunku \ref{fig:exp1-populacja} przedstawione zostały przebiegi czasowe bilansu pszczół w jednym z przeprowadzonych eksperymentów. Wyznaczono je osobno dla poszczególnych tuneli, oraz łącznie dla całego urządzenia, poprzez sumowanie łącznej liczby osobników, które do danej chwili $t$ weszły do ula, minus te, które go opuściły daną ścieżką.
\begin{figure}[htb]
    \centering
    \includegraphics{tex/img/exp1-populacja.pdf}
    \caption{Bilans pszczół w poszczególnych tunelach podczas jednego z eksperymentów.}
    \label{fig:exp1-populacja}
\end{figure}
Można zaobserwować, że na początku eksperymentu owady prawie wyłącznie wchodziły do ula. Jest to rezultat nasilonego powrotu osobników, które opuściły gniazdo podczas montażu czujnika w grupowym instynkcie obronnym.
Po kilkunastu minutach tendencja ta została zmieniona, a pszczoły zaczęły intensywnie opuszczać ul -- prawdopodobnie wracając do pracy po jej zaburzeniu.
Warto również zauważyć, że owady miały swój ulubiony tunel, którym wchodziły do ula (tunel 0), oraz ulubiony tunel wyjściowy (tunel 6).

Należy zwrócić uwagę na możliwe problemy wynikające z przedstawionej metody prowadzenia anotacji danych.
W przeprowadzonym procesie, oznaczano na podstawie nagrania wideo chwile, w których pszczoła pojawiała się na wlocie tunelu.
W ogólności, jest to dość dobre przybliżenie chwili jej przejścia przez czujnik pojemnościowy, bez względu na kilka centymetrów dzielące koniec tunelu od położenia czujnika -- pszczoły zazwyczaj poruszają się ze znaczną szybkością.
Niestety, mogą wystąpić sytuacje szczególne czterech rodzajów:
\begin{enumerate}
    \item Pszczoła spoza ula wlatuje do tunelu, zatrzymuje się przed dotarciem do czujnika, czeka dłuższy czas, i dopiero przelatuje przez bramkę czujnika. W tym przypadku generowana jest etykieta wejścia znacznie wyprzedzająca moment faktycznego pojawienia się impulsu na sygnale bramki;
    \item Pszczoła spoza ula wlatuje do tunelu, a następnie zawraca przed dotarcie do czujnika i opuszcza tunel. W tym przypadku generowane są dwie etykiety: wejście i wyjście, pomimo braku faktycznego wykrycia pszczoły przez bramkę;
    \item Sytuacja symetryczna dla 1: pszczoła z wewnątrz ula przechodzi przez bramkę czujnika, i odczekuje dłuższy czas przed opuszczeniem tunelu. W tym przypadku generowana jest etykieta wyjścia znacznie opóźniona względem faktycznego wykrycia;
    \item Sytuacja symetryczna dla 2: pszczoła z wewnątrz ula przechodzi przez bramkę czujnika, a następnie zawraca przed totarciem do końca tunelu, i przechodząc ponownie przez czujnik wraca do ula. W tym przypadku nie zostanie wygenerowana żadna etykieta, a na sygnale wyjściowym bramki pojawiają się dwa impulsy: wyjście i wejście.
\end{enumerate}
Przeprowadzone zostało porównanie „na oko”, na podstawie którego stwierdzono, że błędy rodzajów 1, 3 oraz 4 nie występują często (nie natrafiono na ich ślady), natomiast czasem zdarzają się błędy rodzaju 2 (pszczoły zawracające przed dotarciem do bramki -- dwie nadmiarowe etykiety).
Nie została wyznaczona konkretna częstotliwość pojawiania się błędów, jednak ustalono, że zachodzą one na tyle rzadko, że nie będą stanowić problemu przy zastosowaniu opracowanej anotacji do oceny działania rozwijanych algorytmów.
Gdyby planowane było stosowanie metod opartych o machine learning, nawet bardzo rzadkie występowanie błędnych etykiet mogłoby znacząco pogorszyć działanie algorytmów, natomiast w przypadku niniejszej pracy, spowoduje ono jedynie lekkie pogorszenie oceny każdego z proponowanych rozwiązań.
Kara ta będzie jednak równa dla każdego z testowanych algorytmów, ponieważ każdy z nich otrzyma tyle samo przykładów testowych z błędnie oznaczonymi rozwiązaniami.

Na podstawie zebranych w niniejszych badaniach przebiegów można jasno stwierdzić, że zaprojektowane czujniki pojemnościowe sprawdzają się we współpracy z żywymi pszczołami poruszającymi się samodzielnie przez tunele urządzenia.
Zebrano zbiór danych, na którym możliwe będzie testowanie metod detekcji impulsów wywoływanych przez owady pojawiające się w bramkach czujników -- możliwe było zatem przejście do projektowania algorytmów przetwarzania sygnałów, opisanych w kolejnych sekcjach.


\subsection{Wstępne przetwarzanie danych}

Niezależnie od stosowanej później metody detekcji, surowe dane z czujnika muszą być wstępnie przygotowane.
Każdy z sygnałów przejdzie dwa kroki przetwarzania: filtr uśredniający oraz usuwanie trendu.
W kolejnych sekcjach opisano te kroki, rozważając je dla sygnału pojedynczego tunelu urządzenia -- dla uproszczenia zapisu pominięto we wzorach indeks czujnika $i$ -- ten sam wzór stosuje się dla każdego z sygnałów $x_i(k)$, gdzie $i=0\dots7$. Ponadto, ściśle podkreślona zostanie ich dyskretna natura, stąd indeksowanie numerem próbki $k$, a nie chwilą czasu ciągłego $t$.

  \subsubsection{Filtr uśredniający}

Sygnały wyjściowe czujników pojemnościowych obarczone są znacznym szumem, widocznym wyraźnie m.in. na rysunku \ref{fig:filtr-brak}. 
Oprócz tego, przyjmują one jedynie wartości całkowite, przy czym amplituda impulsów nie przekracza zwykle 10 -- a co za tym idzie -- ich rozdzielczość pozostawia wiele do życzenia.
Oba te problemy rozwiązywane są poprzez nałożenie na sygnały filtra średniej kroczącej.
Zdecydowano się na ten rodzaj filtra, ponieważ jest optymalny na potrzeby niniejszego zastosowania, a przy tym jest jednym z najprostszych i najłatwiejszych w implementacji filtrów cyfrowych \cite{smith1997}.
Filtr ten zaimplementowano zgodnie z przedstawionym poniżej wzorem:
\begin{equation}
    x_\mathrm{f}(k) = \frac{1}{N}\sum^{N-1}_{i=0}{x(k-i)},
\end{equation}
gdzie $x_\mathrm{f}(k)$ oznacza przefiltrowaną wersję sygnału $x(k)$, a $N$ to wybrana długość okna -- parametr filtra.
Na rysunku \ref{fig:filtr} przedstawione zostało zestawienie oryginalnego, nieprzefiltrowanego sygnału, z wynikiem działania filtra średniej kroczącej o przykładowej długości okna $N=50$.
\begin{figure}[htb]
    \centering
    \begin{subfigure}{\textwidth}
        \centering
        \caption{Przebieg sygnału oryginalnego}
        \includegraphics{tex/img/filtr-brak.pdf}
        \label{fig:filtr-brak}
    \end{subfigure}
    \begin{subfigure}{\textwidth}
        \centering
        \caption{Sygnał filtrowany filtrem średniej kroczącej o oknie długości $N=50$}
        \includegraphics{tex/img/filtr-jest.pdf}
        \label{fig:filtr-jest}
    \end{subfigure}
    \caption{Porównanie sygnału oryginalnego z przefiltrowanym filtrem średniej kroczącej.}
    \label{fig:filtr}
\end{figure}
Widoczne są pożądane efekty pracy filtra:
\begin{itemize}
    \item Po pierwsze, zlikwidował on prawie zupełnie sumy obecne w oryginalnych danych;
    \item Ponadto, dzięki uśrednieniu próbek, zbiór wartości sygnału znacznie się powiększył -- zwiększona została znacząco jego rozdzielczość.
\end{itemize}
W wyniku obu tych zjawisk, przebieg wyjścia czujnika stał się znacznie łagodny, a impuls wygenerowany przez ruch pszczoły w tunelu jest bardzo wyraźnie widoczny.
Można stwierdzić, że zaprojektowana metoda filtrowania danych sprawdza się w przypadku niniejszego zastosowania.
Warto zwrócić uwagę, że filtrowanie sygnału wpłynęło na znaczące zmniejszenie amplitudy prezentowanego impulsu -- stanowi to pewien koszt filtrowania.
Dopóki jednak redukcja amplitudy nie zrównuje jej z pozostającym po filtrowaniu szumem, wzory generowane przez pszczoły pozostaną odróżnialne i będzie możliwe ich wykrycie.
Inną wadą stosowania filtrów może być wprowadzanie przez nie opóźnienia sygnału, jednak w przypadku projektowanego systemu nie stanowi to zagrożenia -- nie zawiera on sprzężeń zwrotnych, które mogłyby się zdestabilizować, a poza tym opóźnienie w zliczeniu pszczoły, nawet wynoszące wiele sekund, zupełnie nie ma znaczenia dla prawidłowości ogólnego szacunku populacji w skali dnia.

Istotne jest odpowiednie dobranie długości okna $N$, aby zastosowany filtr skutecznie tłumił szumy sygnału, ale przy tym nie redukował zbytnio amplitudy odpowiedzi na ruch pszczoły.
Na rysunku \ref{fig:filtr-dlugosc} przedstawione zostało porównanie tego samego fragmentu przebiegu poddanego filtracji z różnymi wartościami $N$: 5, 50 oraz 500.
\begin{figure}[htb]
    \centering
    \begin{subfigure}{\textwidth}
        \centering
        \caption{Sygnał filtrowany filtrem średniej kroczącej o oknie długości $N=5$}
        \includegraphics{tex/img/filtr-slaby.pdf}
        \label{fig:filtr-dlugosc-5}
    \end{subfigure}
    \begin{subfigure}{\textwidth}
        \centering
        \caption{Sygnał filtrowany filtrem średniej kroczącej o oknie długości $N=50$}
        \includegraphics{tex/img/filtr-dobry.pdf}
        \label{fig:filtr-dlugosc-50}
    \end{subfigure}
    \begin{subfigure}{\textwidth}
        \centering
        \caption{Sygnał filtrowany filtrem średniej kroczącej o oknie długości $N=500$}
        \includegraphics{tex/img/filtr-zasilny.pdf}
        \label{fig:filtr-dlugosc-500}
    \end{subfigure}
    \caption{Porównanie działania filtrów o długościach okna: (a) $N=5$, (b) $N=50$, (c) $N=500$.}
    \label{fig:filtr-dlugosc}
\end{figure}
Na ostatnim z prezentowanych wykresów (rysunek \ref{fig:filtr-dlugosc-500}), widoczne są efekty zbyt długiego okna: szum został skutecznie usunięty, jednak praktycznie zupełnie pozbyto się również impulsów generowanych przez pszczoły.
Filtr o $N=500$ jest znacząco zbyt mocny.
Z kolei na pierwszym z wykresów (rysunek \ref{fig:filtr-dlugosc-5}), impulsy pozostają wyraźnie widoczne, ale redukcja szumu nie jest wystarczająca.
Dobrą wartością długości okna filtra okazuje się być prezentowane wcześniej $N=50$ (wykres na rysunku \ref{fig:filtr-dlugosc-50}): szumy oryginalnego sygnału są w znacznej mierze usunięte, natomiast odpowiedź sygnału na obecność pszczoły w tunelu pozostaje ostra i wyraźnie widoczna.
Konkretne wartości $N$ zostaną wybrane na etapie optymalizacji parametrów poszczególnych algorytmów detekcji, jednak już na tym etapie można stwierdzić, że wartości optymalnej należy szukać w pobliżu $N=50$.


\FloatBarrier
  \subsubsection{Usuwanie trendu}

Może tutaj porównanie metod... bo w sumie testowałem różne

\begin{equation}
    y(k) = x_\mathrm{f}(k) - \text{median}\left(\begin{bmatrix} 
    x_\mathrm{f}(k) \\
    x_\mathrm{f}(k-1) \\ 
    \vdots \\ 
    x_\mathrm{f}(k-N) 
\end{bmatrix}
\right)
\end{equation}


  
\subsection{Opis proponowanych algorytmów}
  \subsubsection{Automat stanowy}
  \subsubsection{Korelacja wzajemna z wzorcem}
  \subsubsection{Klasyfikator na cechach sygnału}
\subsection{Wyniki testów}
\subsection{Wybór algorytmu}
\subsection{Implementacja algorytmu na sprzęcie}