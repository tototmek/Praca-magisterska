\clearpage
\section{Algorytm detekcji} \label{chapter:algorytm}

Opisany w poprzedniej części pracy system czujników generuje sygnały wyjściowe, na przebiegach czasowych których widoczne są impulsy wywołane ruchem pszczół.
W celu umożliwienia wykorzystania stworzonego urządzenia do automatycznego zliczania pszczół w ulu, konieczne jest zaprojektowanie algorytmu wykrywającego te impulsy.
Przyjęte zostały następujące założenia:
\begin{enumerate}
    \item Algorytm ma być uruchomiony na mikrokontrolerze wchodzącym w skład urządzenia -- co pozwoli uniknąć konieczności rozszerzenia systemu o dodatkowy węzeł, którego zadaniem byłoby jedynie wykonywanie obliczeń.
    Wymagać to będzie niskiej złożoności pamięciowej algorytmu oraz niewielkiego narzutu obliczeniowego.
    \item Algorytm ma być wykonywany \textit{on line} -- w każdej iteracji działania urządzenia, lub na buforze z maks. kilku sekund. W ten sposób zagwarantowane będą bieżące dane, oraz brak przerw w akwizycji sygnału z czujników pojemnościowych.
\end{enumerate}


\subsection{Zbiór danych przykładowych}

Pierwszym krokiem, zanim rozpoczęto pracę nad projektem algorytmu, było zebranie dużej ilości przykładowych danych z pracy urządzenia w warunkach rzeczywistych.
Są one niezwykle ważne, pozwolą bowiem do oceny proponowanych w kolejnych sekcjach algorytmów i ich konfiguracji -- posłużą do przeprowadzenia testów \textit{off line}.

  \subsubsection{Testy urządzenia w pasiece}

W celu uzyskania reprezentatywnych danych z pracy urządzenia w środowisku rzeczywistym przeprowadzone zostały eksperymenty w dwóch niezależnych pasiekach, w różnych tygodniach, przy innych warunkach atmosferycznych i aktywności pszczół.
Testy prowadzono według następującej metody:
\begin{enumerate}
    \item Wybierano ul, który zostanie wykorzystany do eksperymentu. W obu przypadkach kierowano się największą aktywnością pszczół w całej pasiece, by w zebranych przebiegach wystąpiło jak najwięcej przejść owadów przez czujnik;
    \item Na wylotku wybranego ula umieszczano urządzenie. Zastosowano dwie alternatywne jego pozycje: wewnątrz ula -- zamocowane do dennicy, a także na zewnątrz ula na półce przed wylotkiem.
    Pszczelarze w prowadzonych rozmowach sugerowali, że przy drugiej z wymienionych metod pszczoły mogą być niechętne do wchodzenia do gniazda, jednak w eksperymencie nie zaobserwowano tego zjawiska.
    Wszelkie szczeliny pozostające między urządzeniem a ścianami ula zatkano, aby uniemożliwić pszczołom ominięcie tuneli czujnika.
    \item W obu eksperymentach odczekano około 30 minut, aby pszczoły poruszone zamieszaniem przy montażu urządzenia uspokoiły się, a następnie aby przyzwyczaiły się do nowego obiektu na wylotku.
    \item Na statywie przed ulem umieszczono kamerę wideo, nagrywającą obraz urządzenia. Kadr dobrany został tak, by wyraźnie widoczne były wloty wszystkich tuneli urządzenia.
    Nagranie to będzie później wykorzystane do anotacji danych -- ręczne opisu chwil, w których pszczoły wchodziły i wychodziły.
    \item Urządzenie łączono z portem USB laptopa, aby zapewnić przesył kolejnych próbek sygnałów wyjściowych każdego z czujników pojemnościowych.
    Uruchamiano stworzony uprzednio skrypt Python, zapisujący otrzymywane dane do pliku CSV.
    \item W chwili uruchomienia w komputerze akwizycji sygnału, wykonywano przed kamerą klaśnięcie, co pozwoli na późniejszą synchronizację nagrania z zebranymi danymi.
\end{enumerate}
Na rysunku \ref{fig:exp2} przedstawiona jest fotografia prezentująca jeden z przeprowadzonych eksperymentów, natomiast na rysunku \ref{fig:exp1} przedstawiono alternatywne miejsce montażu urządzenia (od środka ula).
W tabeli \ref{tab:test1} podsumowano informacje o przeprowadzonych eksperymentach.
\begin{table}[H]
\centering
\caption{Zbiór danych przykładowych -- podsumowanie eksperymentów}
\begin{tabular}{l|l|l|l}
                       & \multicolumn{1}{c|}{\textbf{Eksperyment 1}} & \multicolumn{1}{c|}{\textbf{Eksperyment 2}} &                                                \\ 
\cline{1-3}
\textbf{Data}          & 20 lipca 2025                               & 26 lipca 2025                               &                                                \\
\textbf{Lokalizacja}   & gmina Miastkowo                             & gmina Choroszcz                             & \multicolumn{1}{c}{\textit{\textbf{Łącznie}}}  \\ 
\cline{4-4}
\textbf{Czas trwania}  & ok. 45 min                                  & ok. 30 min                                  & ok. 1 h 15 min                                 \\
\textbf{Liczba próbek} & 274214                                      & 182435                                      & 456649                                        
\end{tabular}
\label{tab:test1}
\end{table}

Przeprowadzone eksperymenty stanowiły również pierwszą weryfikację działania urządzenia w warunkach rzeczywistych -- przy współpracy z żywymi pszczołami.
Fragment zebranych danych zwizualizowano na wykresie (rysunek \ref{fig:exp1-data}), w celu sprawdzenia czy zawiera prawidłowo wyglądające impulsy wywoływane ruchem pszczół.
Stwierdzono, że system działa prawidłowo: pomimo występującego szumu widoczne są wzory bardzo podobne do wzorcowych przebiegów; warto porównać zaprezentowane przebiegi z tymi na rysunku \ref{fig:campbell-symulacja-wynik}.
\begin{figure}[H]
    \centering
    \includegraphics{tex/img/wizualizacja-testu-z-pasieki.pdf}
    \caption{Przebieg sygnału $x_0(t)$ (wyjścia pierwszego czujnika pojemnościowego) w pierwszych kilkudziesięciu sekundach trwania eksperymentu. Przebieg filtrowany został wygenerowany poprzez nałożenie na dane oryginalne filtru średniej kroczącej o długości okna równej 50.}
    \label{fig:exp1-data}
\end{figure}
\begin{figure}[p]
    \centering
    \includegraphics[width=0.9\linewidth]{tex/img/8-eksperyment-2-downscaled.jpg}
    \caption{Zdjęcie jednego z eksperymentów. Widoczne jest urządzenie zamontowane na wylotku ula, oraz kamera nagrywające jego wylot. Znajdujący się na obrazie fioletowy przewód łączy czujnik z komputerem zapisującym dane.}
    \label{fig:exp2}
\end{figure}
\begin{figure}[p]
    \centering
    \includegraphics[width=0.9\linewidth]{tex/img/7-eksperyment-1.JPG}
    \caption{Zdjęcie prezentujące montaż urządzenia podczas drugiego eksperymentu. Jest ono przybite do dennicy -- pozostałą część ula nastawia się na widoczny na obrazie element.}
    \label{fig:exp1}
\end{figure}

  
  \subsubsection{Ręczna anotacja danych}
\subsection{Wstępne przetwarzanie danych}
  \subsubsection{Filtr uśredniający}
  \subsubsection{Usuwanie trendu}
\subsection{Opis proponowanych algorytmów}
  \subsubsection{Automat stanowy}
  \subsubsection{Korelacja wzajemna z wzorcem}
  \subsubsection{Klasyfikator na cechach sygnału}
\subsection{Wyniki testów}
\subsection{Wybór algorytmu}
\subsection{Implementacja algorytmu na sprzęcie}