\clearpage
\section{Wprowadzenie}
\subsection{Wstęp teoretyczny}

Pszczelarstwo jest zajęciem wymagającym wiele pracy, wiedzy i~doświadczenia.
W dzisiejszych czasach, pszczelarze wspierani są różnorodnym sprzętem, umożliwiającym skuteczniejszą opiekę nad ulami, oraz wydajną produkcję miodu.
Ponadto, przez ostatnie lata rozwijane były nowoczesne technologie, które, wykorzystując elektronikę, pozwalają dogłębnie monitorować stan pszczelich kolonii.
Rosnące w~popularność zaawansowane czujniki zapewniają pszczelarzom precyzyjne dane o~zdrowiu pszczół, pozwalające na skuteczniejsze zarządzanie pasieką -- z~pozytywnymi skutkami zarówno dla owadów, jak i~efektywności samego biznesu.
Stosowane szeroko rozwiązania pozwalają, jak opisują Hadjur \cite{Hadjur2022} i~Danieli \cite{Danieli2024} wraz z~zespołami, na monitorowanie takich parametrów jak: temperatura, wilgotność i~stężenie $\text{CO}_2$ wewnątrz ula -- których nieprawidłowe poziomy mogą powodować choroby; waga całego gniazda, świadcząca o~łącznej ilości pszczół, zebranego miodu, i~innych zasobów.
Stosuje się również czujniki akustyczne, mierzące częstotliwość, intensywność i~barwę wibracji, pozwalające na wykrywanie podwyższonego stresu pszczół, a~także początków rojenia się.

Jednym z~zadań realizowanym przez systemy elektroniczne, wymienianym przez Meiklego i~Holsta \cite{Meikle2014}, jest zliczanie pszczół wchodzących do ula i~opuszczających go -- co pozwala szacować chwilowe natężenie ruchu zbieraczek.
Metryka ta zapewnia informacje odnośnie siły i~struktury wiekowej kolonii, a~także zapotrzebowania i~dostępności pożywienia \cite{McLellan1977}.
Zliczanie pszczół ma również znaczenie podczas prowadzenia badań naukowych, które ich dotyczą -- między innymi prace Kolmesa i~Sama \cite{Kolmes1990}, czy Corbeta z~zespołem \cite{Corbet1993}.

\subsection{Stosowane metody automatycznego zliczania pszczół}

Potrzeba liczenia pszczół prowadziła przez lata do rozwoju metod automatyzujących to zadanie.
Przegląd kolejnych pojawiających się rozwiązań \cite{Odemer2021} otwiera ponad stuletnia praca Lundiego z~1925 roku -- urządzenie elektromechaniczne, w~którym owady przechodzące przez system zapadek aktywowały styki, wysyłając tym samym sygnał do licznika.
Podejście to było awaryjne i~wymagało częstego czyszczenia z~nanoszonego pyłku, jednak pozwoliło autorowi na realizację badań o~życiu pszczół, i~położyło podwaliny pod dalszy rozwój podobnych technik \cite{Lundie1925}.
W późniejszych latach powstawały systemy zliczania pszczół oparte na układach optycznych. 
Wykorzystywano fotokomórki na światło podczerwone, którego promień, niewidzialny dla pszczół, był przez nie przecinany, co generowało sygnał.
Do pierwszych prac bazujących na tej technice należy dzieło Ericksona wraz z~zespołem z~roku 1975 \cite{Erickson1975}.
Globalny rozwój technologii umożliwił dalsze rozpowszechnianie rozwiązań opartych na optyce: systemy wykorzystujące technologię LED zaprezentowali m.in. Pešović \cite{Pesovic2017} i~Jiang \cite{Jiang2016} wraz z~zespołami, a~także, w~ostatnich latach, Cunha z~zespołem \cite{Cunha2020}.
Autorzy zwrócili uwagę na znaczącą wadę takiego podejścia -- zbierający się w~torze optycznym pyłek po pewnym czasie uniemożliwiał poprawne działanie systemu, wymuszając regularne czyszczenie wnętrz urządzeń.
Pomimo tego ograniczenia, urządzenia zliczające pszczoły oparte na technologii LED są dzisiaj dostępne komercyjnie \cite{sklep}.

W ostatnich latach, w~literaturze pojawia się coraz więcej systemów zliczających pszczoły opartych na zaawansowanych algorytmach przetwarzania obrazu i~wizji komputerowej.
Narzędzie wykrywające pszczoły w~obrazie wykorzystujące konwolucyjne sieci neuronowe zostało zaprezentowane przez Tauscha wraz z~zespołem \cite{Tausch2020}.
Takie rozwiązania są znacznie bardziej wymagające obliczeniowo, jednak mogą pozwalać na realizację dodatkowych funkcji -- system zaprezentowany przez Bilika wraz z~zespołem \cite{Bilik2021} umożliwia wykrycie roztoczy \textit{Varroa} na pszczołach, a~rozwiązanie Dunkera wraz z~zespołem \cite{Dunker2021} pozwala na ocenę ilości pyłku transportowanego przez każdą z~pszczół.

Podejściem, które, mimo swej prostoty, wciąż nie zostało szeroko przyjęte jest zastosowanie do wykrywania pszczół czujników pojemnościowych.
Wykorzystanie różnicy pojemności elektrycznej kondensatora wywołanej ruchem pszczoły zostało pierwszy raz zastosowane w~pracy Campbell wraz z~zespołem \cite{Campbell2005}, która zaprojektowała czujnik składający się z~pierścieniowych okładek, między którymi poruszają się owady.
Zmodyfikowana wersja podobnego systemu została zaprezentowana przez Perraulta i~Teachmana \cite{Perrault2016}, którzy zwrócili uwagę na szereg zalet tego rozwiązania: czujnik pojemnościowy nie jest podatny na zabrudzenia i~nie wymaga częstego czyszczenia; ponadto, jest całkowicie nieszkodliwy dla pszczół.
Zwrócono uwagę na podatność systemu na błąd wynikający z~grupowania się pszczół.
Problem ten zaadresował Bermig wraz z~zespołem \cite{Bermig2020}, konstruując urządzenie \textit{BeeCheck} posiadające siedem bramek pojemnościowych w~torze ruchu pszczół.
W ostatnim czasie popularność pojemnościowych liczników pszczół zaczęła rosnąć, pojawił się nawet internetowy artykuł instruktażowy opisujący, jak wykonać podobne urządzenie samodzielnie \cite{mathematastic}.



\subsection{Cel i~zakres pracy}
Celem niniejszej pracy jest stworzenie urządzenia zliczającego pszczoły w~ulu opartego o~zestaw czujników pojemnościowych.
Ma ono być montowane na wylotku (szczelinie, przez którą pszczoły wchodzą do ula i~opuszczają go) tak, by cały ruch owadów odbywał się poprzez czujniki.
Jego zadaniem jest wykrycie każdej pojawiającej się pszczoły, określenie chwili jej przejścia, a~także kierunku jej ruchu (wejście/wyjście) -- w~celu umożliwienia zliczania pszczół aktualnie znajdujących się w~ulu.

Na pracę składać się będą rozwiązania dwóch głównych problemów:
\begin{enumerate}
    \item zaprojektowanie czujników pojemnościowych i~metody akwizycji ich sygnałów, a~także konstrukcja urządzenia wyposażonego w~ich szereg,
    \item opracowanie algorytmu detekcji, umożliwiającego wykrywanie pszczół na podstawie wyjść poszczególnych czujników pojemnościowych, a~także jego implementacja na platformie sprzętowej urządzenia.
\end{enumerate}
Tworzony system zostanie kompleksowo przetestowany, zarówno w~warunkach laboratoryjnych, jak i~w pracy w~środowisku rzeczywistym, by zweryfikować skuteczność zliczania pszczół i~sprawdzić jego podatność na błędy.
